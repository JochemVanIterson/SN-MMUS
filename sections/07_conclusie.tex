\section{Conclusie en reflectie}
\lipsum[7]

% In dit hoofdstuk vindt de terugkoppeling naar zowel de uitvoering van het werk (hfs 4) en het oorspronkelijke doel (hfs 1) plaats. Wanneer alle onderdelen van de paper - doel, vraagstelling, methode, uitvoering - goed op elkaar aansluiten - een goede 'logical construct' -, zal er altijd een zinvolle conclusie te maken zijn. Ook wanneer de research, tegen de verwachting in, niet veel oplevert is dat van belang als conclusie: wanneer de constructie van het geheel goed is geweest, zal iemand die zich met hetzelfde onderwerp gaat bezighouden de research kunnen volgen en een alternatieve aanpak kunnen ontwikkelen. In dit hoofdstuk komen dan ook aanbevelingen aan de orde: de losse draden die in toekomstige projecten opgepakt kunnen worden. Een paper of thesis bevat naast bovenstaande hoofdstukken altijd een (engelstalige) samenvatting (abstract) aan het begin, en een standaard opgemaakt literatuurlijst (Harvard style) aan het einde. Bij een meer uitgebreide paper of thesis kan allerlei ondersteunend materiaal in bijlages toegevoegd
