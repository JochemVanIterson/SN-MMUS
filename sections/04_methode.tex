\section{Methode}
% De hoofdstukken introductie en critical review leiden in combinatie tot een vraagstelling. Deze vraagstelling moet goed in lijn liggen met het doel van het research. De vraagstelling en daarmee samenhangende definities bakenen het research verder af. In dit hoofdstuk wordt vervolgens de methode beschreven waarmee deze vraag wordt beantwoord. Van belang is dat de methode goed moet aansluiten op het doel en de vraagstelling, zodat er een logische samenhang is. Uitgangspunt is vaak een model - een al bestaand model of eigen voorstel - dat gebruikt wordt als 'bril' om naar het onderwerp te kijken. Er wordt een keus gemaakt uit een aantal methodes: participant observation, case studies, interviews, implementeren van in literatuur gevonden theorie, experimenten in de eigen praktijk, enzovoort. Omdat het hier om practice based research gaat is de eigen praktijk altijd leidend bij het kiezen van een methode. Het is belangrijk dat de methode goed aansluit bij het doel en de vraagstelling van het stuk.

Het onderzoek van dit supportive narrative zal bestaan uit literatuur onderzoek. Dit zal worden gedaan door middel van het lezen van artikelen, papers en het bekijken van video's.
Het onderzoek zal zich richten op het zoeken van argumentaties voor de aannames die ik heb gedaan bij het ontwerpen van de NichePlayer. Deze aannames zijn:

\begin{itemize}
  \item Door het vernieuwde luistergedrag binnen streamingdiensten heeft het album als distributievorm zijn individuele waarde verloren
  \item De waarde zit op dit moment niet meer in de muziek, maar het platform die toegang biedt tot de (haast onbeperkte) content
  \item De waarde van muziek zit in de community die wordt gevormd rondom de muziek
  \item Doordat beginnende artiesten op streamingdiensten direct concurreren met artiesten op professioneel niveau is het moeilijk om door te breken. Dit is slecht voor de diversiteit binnen de muziekindustrie
  \item Hoewel het steeds makkelijker is om muziek te uploaden, zorgt dit er ook voor dat de muziek kwijt raakt in het grote geheel
\end{itemize}

De reden dat ik dit onderzoek zo opbouw is omdat ik het belangrijk vind om de aannames die ik heb gedaan te onderbouwen. Ik wil niet dat de NichePlayer een product wordt dat gebaseerd is op aannames, maar op feiten. Hiermee kan het makkelijker aansluiting vinden in de bestaande digitale muziek industrie.

Naast het literatuur onderzoek ga ik ook zelf het proces van muziek publishing doorlopen. Dit zal ik doen door middel van het maken van een EP (met bestaande content) en deze te publiceren op Spotify. Het doel hiervan is om de aannames die ik van anderen heb overgenomen zelf ook te ervaren.

De uitkomst van het onderzoek zal verwerkt worden in het businessplan en het prototype van het project. Het businessplan moet gebruikt kunnen worden bij het pitchen van het project. Het prototype zal gebruikt worden om het product te testen bij de doelgroep.

\subsection{Project}
De resultaten van dit (literatuur) onderzoek zullen worden verwerkt in het ontwikkelen van de NichePlayer. Het ontwikkelen hiervan is al enkele jaren gaande. Dit komt omdat ik vaak opnieuw ben begonnen, het is een iteratief proces. Er wordt veel geïnnoveerd in de software industrie waardoor er na iedere versie wel weer nieuwe technieken beschikbaar bleken te zijn die het product verder konden verbeteren.

Daarnaast werk ik binnen de grote iteraties zelf als rapid prototyper. Ik werk steeds aan kleine stukjes om te kijken of een bepaalde techniek werkt, of interessante resultaten geeft. Niet alle prototypes zijn nuttig voor het eindresultaat, maar zijn wel een belangrijk onderdeel van mijn ontwikkelproces.

De globale lijnen van een iteratie zijn vaak wel linear. Veel onderdelen van een systeem bouwen op elkaar voort, waardoor het niet mogelijk is om een onderdeel te ontwikkelen zonder dat de rest van het systeem al bestaat. Hierbij is het niet altijd mogelijk een correcte inschatting te maken van tijd die nodig is voor een iteratie. Dit komt omdat er vaak onverwachte problemen opduiken die het ontwikkelen vertragen.

Een probleem bij op deze manier werken is dat het moeilijk is om te bepalen wanneer een iteratie af is. Er zijn altijd meer functies die uitgewerkt kunnen worden, of nieuwe technieken die toegepast kunnen worden. Vaak denk ik dat een iteratie de definitieve versie zal zijn, maar na driekwart van het project te hebben ontwikkeld stuit ik toch op een foute ontwerpbeslissing. Dit is een van de redenen dat ik vaak opnieuw ben begonnen.

Een nieuwe ontwikkelmethode die ik dit project ga uitproberen is het niet werken aan het product. Wanneer ik prototypend aan een project werk blijk ik nooit tot de staat van een definitief product te komen. Dit komt omdat ik steeds weer opnieuw wil beginnen omdat fundamentele onderdelen van het systeem beter zouden kunnen. Door eerst te prototypen in meerdere andere projecten, en daarna de resultaten hiervan samen te voegen tot het eindproduct kan ik doelgerichter te werk gaan tijdens het samenvoegen. Ik kan hiermee nog steeds werken als rapid prototyper, en de resultaten hiervan gebruiken in het eindproduct.

\subsection{Reverse-linear}
Soms is het lastig een richting te vinden wanneer ik werk, zeker als rapid prototyper. Ik heb tijdens een meeting met mijn supervisor de tip gekregen om éérst het eindresultaat in de vorm van een Wizard of Ozz te maken, en daar vervolgens naartoe te werken. Deze manier van werken noem ik vanaf dit moment reverse-linear. Ik begin met het eindresultaat, en werk vervolgens daar naar toe. Hierdoor heeft het schrijven en ontwikkelen gelijk richting. Het is een manier van werken die ik nog niet eerder heb toegepast, maar ik ben erg benieuwd naar de resultaten.

% \subsection*{OUTLINE}
% In dit hoofdstuk zal de methode worden beschreven van het onderzoek. Het hoofdonderzoek van dit SN is bedoeld om de bovengenoemde aannames te onderbouwen. Dit zal worden gedaan door middel van literatuur en deskresearch en het houden van kwalitatieve interviews met artiesten, producers en luisteraars. De interviews zullen worden gevoerd met mensen die de NichePlayer zouden kunnen gebruiken, zowel artiest als gebruiker. Het doel is om te kijken of de NichePlayer een toegevoegde waarde heeft, en wat de plek is in de wereld van muziek distributie.

% Daarnaast zal ik zelf het process van muziek publishing gaan doorlopen om mijn aannames bij het publiceren van muziek te testen.
