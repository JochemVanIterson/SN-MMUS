\section{Methode}
% De hoofdstukken introductie en critical review leiden in combinatie tot een vraagstelling. Deze vraagstelling moet goed in lijn liggen met het doel van het research. De vraagstelling en daarmee samenhangende definities bakenen het research verder af. In dit hoofdstuk wordt vervolgens de methode beschreven waarmee deze vraag wordt beantwoord. Van belang is dat de methode goed moet aansluiten op het doel en de vraagstelling, zodat er een logische samenhang is. Uitgangspunt is vaak een model - een al bestaand model of eigen voorstel - dat gebruikt wordt als 'bril' om naar het onderwerp te kijken. Er wordt een keus gemaakt uit een aantal methodes: participant observation, case studies, interviews, implementeren van in literatuur gevonden theorie, experimenten in de eigen praktijk, enzovoort. Omdat het hier om practice based research gaat is de eigen praktijk altijd leidend bij het kiezen van een methode. Het is belangrijk dat de methode goed aansluit bij het doel en de vraagstelling van het stuk.

\subsection{Methode Onderzoek}

Het onderzoek in dit Supportive Narrative zal bestaan uit literatuuronderzoek. Dit zal worden gedaan door middel van het lezen van artikelen, papers en het bekijken van video's. Het onderzoek zal zich richten op het zoeken naar argumentaties voor de aannames die ik in het verleden heb gedaan bij het ontwerpen van de NichePlayer. Deze aannames zijn:

\begin{itemize}
  \item Door het vernieuwde luistergedrag binnen streamingdiensten heeft het album als distributievorm zijn individuele waarde verloren.
  \item De waarde zit op dit moment niet meer in de muziek, maar in het platform dat toegang biedt tot de voor de luisteraar haast onbeperkte hoeveelheid aan content.
  \item De waarde van muziek zit in de community die wordt gevormd rondom de muziek.
  \item Doordat beginnende artiesten op streamingdiensten direct concurreren met artiesten op professioneel niveau is het moeilijk om door te breken. Dit is slecht voor de diversiteit en innovatie binnen de muziekindustrie.
  \item Hoewel het steeds makkelijker is om muziek te uploaden, zorgt dit er ook voor dat de muziek kwijtraakt in het grote geheel.
\end{itemize}

De reden dat ik dit onderzoek zo opbouw is omdat ik het belangrijk vind om de aannames die ik heb gedaan te onderbouwen. Ik wil niet dat de NichePlayer een product wordt dat gebaseerd is op ongefundeerde aannames, maar op feiten en correcte redenatie. Hiermee kan het makkelijker aansluiting vinden in de bestaande digitale muziekindustrie.

De uitkomst van het onderzoek zal verwerkt worden in het businessplan en het prototype van het project. Het businessplan moet gebruikt kunnen worden bij het pitchen van het project. Het prototype zal gebruikt worden om het product te testen bij de doelgroep.

\subsection{Methode Project}
De resultaten van dit (literatuur) onderzoek zullen worden verwerkt in het ontwikkelen van de NichePlayer. Het ontwikkelen hiervan is al enkele jaren gaande. Dit komt omdat ik vaak opnieuw ben begonnen, het is een iteratief proces. Er wordt veel geïnnoveerd in de software-industrie waardoor er na iedere versie wel weer nieuwe technieken beschikbaar bleken te zijn die het product verder konden verbeteren. Verbeteren kan zijn op basis van de gebruikerservaring, maar ook op basis van de technische implementatie en toekomstbestendigheid van het systeem.

Daarnaast werk ik binnen de grote iteraties zelf als rapid prototyper. Ik werk steeds aan kleine stukjes om te kijken of een bepaalde techniek werkt, of interessante resultaten geeft. Niet alle prototypes zijn nuttig voor het eindresultaat, maar zijn wel een belangrijk onderdeel van mijn ontwikkelproces.

De globale lijnen van een iteratie zijn vaak wel lineair. Veel onderdelen van een systeem bouwen op elkaar voort, waardoor het niet mogelijk is om een onderdeel te ontwikkelen zonder dat de rest van het systeem al bestaat. Hierbij is het niet altijd mogelijk een correcte inschatting te maken van tijd die nodig is voor een iteratie. Dit komt omdat er vaak onverwachte problemen opduiken die het ontwikkelen vertragen.

Een probleem van lineair werken is dat er pas redelijk laat (zichtbaar of hoorbaar) resultaat is. De kern van de NichePlayer regelt voornamelijk het inloggen van users en het uploaden van media. Hierbij is nog niet duidelijk wat het systeem moet doen aan de frontend, wat inzicht op het proces lastig maakt. Lineair gezien is het logisch dat in een web-systeem eerst media geüpload moet kunnen worden voordat het kan worden afgespeeld, maar dit kost wel veel tijd om te ontwikkelen. Hierdoor is het ook lastig om te bepalen of het systeem wel de goede kant op gaat en of werk en functies nuttig zijn voor het eindproduct.

Een ander probleem bij  deze manier van werken is dat het moeilijk is om te bepalen wanneer een iteratie volledig af is. Er zijn altijd meer functies die uitgewerkt kunnen worden, of nieuwe technieken die kunnen worden toegepast. Vaak denk ik dat een iteratie de definitieve versie zal zijn, maar na driekwart van het project te hebben ontwikkeld stuit ik toch op een foute ontwerpbeslissing eerder in het proces. Dit is een van de redenen dat ik vaak opnieuw ben begonnen.

De belangrijkste voor mij nieuwe ontwikkelmethode die ik in dit project ga uitproberen is het doelbewust niet werken aan het project. Wanneer ik prototypend aan een project werk blijk ik in het verleden nooit in staat geweest om tot een definitief product te komen. Dit komt omdat ik steeds weer opnieuw wil beginnen omdat fundamentele onderdelen van het systeem (bijvoorbeeld gebruikers authenticatie) op nieuwere, stabielere technologie gebouwd zouden kunnen worden. Door eerst te prototypen in (meerdere) andere projecten, en daarna de resultaten hiervan samen te voegen tot het eindproduct kan ik doelgerichter te werk gaan tijdens het samenvoegen. Ik kan hiermee nog steeds werken als rapid prototyper, en de resultaten hiervan gebruiken in het eindproduct. Een ander voordeel hiervan is dat ik code kan hergebruiken in andere projecten, waardoor ook deze profiteren van de resultaten van dit project.

\subsection{Eerst het eindresultaat}
Ik vind het vaak lastig om een richting te vinden en te behouden wanneer ik werk, zeker als rapid prototyper. Ik heb tijdens een meeting met mijn supervisor de tip gekregen om éérst het eindresultaat te schetsen, bijvoorbeeld in de vorm van een Wizard of Oz. Bij deze ontwerpmethode wordt gewerkt met een eindresultaat dat wordt gefaked. Bijvoorbeeld: in plaats van een systeem te ontwikkelen wat exact analyseert wat er in een ruimte gebeurt, is er een persoon die handmatig aan de knoppen van een installatie draait. Hiermee kan het systeem getest worden zonder dat het volledig af is.

Deze methode ga ik gebruiken voor mijn volledige ontwikkelproces van de NichePlayer: ik begin met het beschrijven en schetsen van het eindresultaat, en werk vervolgens daarnaartoe. Hierdoor heeft het schrijven en ontwikkelen gelijk richting. Het is een manier van werken die ik nog niet eerder heb toegepast, maar ik ben erg benieuwd naar de resultaten, en of het mij verder brengt dan bij oude projecten. In het verleden had ik vaak wel een plan, maar dit veranderde constant en was vaak moeilijk uit te leggen aan anderen.

% \subsection*{OUTLINE}
% In dit hoofdstuk zal de methode worden beschreven van het onderzoek. Het hoofdonderzoek van dit SN is bedoeld om de bovengenoemde aannames te onderbouwen. Dit zal worden gedaan door middel van literatuur en deskresearch en het houden van kwalitatieve interviews met artiesten, producers en luisteraars. De interviews zullen worden gevoerd met mensen die de NichePlayer zouden kunnen gebruiken, zowel artiest als gebruiker. Het doel is om te kijken of de NichePlayer een toegevoegde waarde heeft, en wat de plek is in de wereld van muziek distributie.

% Daarnaast zal ik zelf het process van muziek publishing gaan doorlopen om mijn aannames bij het publiceren van muziek te testen.
