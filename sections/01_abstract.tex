% \section*{Abstract}
\begin{abstract}
Dit Supportive Narrative gaat over technologie in het distributieproces van
muziek. Streaming is de afgelopen jaren een belangrijke vorm van muziekconsumptie geworden. Het lijkt echter dat de grote voordelen die artiesten zouden hebben alleen werken bij artiesten die al bekend en rendabel zijn op bestaande diensten. Een oplossing die wordt gegeven is het teruggaan naar fysieke verkoop van muziek. Er zijn verschillende argumenten die dit ondersteunen, welke worden besproken en onderbouwd in dit Supportive Narrative. Verder wordt er een prototype-streamingdienst ontwikkeld die de voordelen van streaming combineert met de voordelen van fysieke verkoop. Het prototype maakt gebruik van NFC om een fysiek artefact te koppelen aan een digitaal product.

In de critical review wordt gekeken naar verschillende geluiddragende media en hoe zij invloed hebben gehad op de muziekindustrie. Hierbij wordt beschreven wat de disruptieve werking is van het medium ten opzichte van zijn voorganger en wat het probleem is wat het medium heeft proberen op te lossen.

Er zijn twee artefacten: een prototype streamingdienst en een uitgeschreven businessplan. Deze twee artefacten bouwen op elkaar voort, en zijn synchroon opgezet en ontwikkeld. Het businessplan bestaat uit het onderbouwen van verschillende aannames die daarmee het businessplan verder ondersteunen. Het prototype is een uitwerking van het businessplan, en is een werkend product. Bij het ontwikkelen van het prototype zijn nieuwe ontwikkelmethoden gebruikt, zoals een combinatie van lineair en iteratief werken, en een manier om uitstelgedrag actief in te zetten.

% In het hoofdstuk uitvoering en resultaten wordt het businessplan verder uiteengezet. Daarnaast wordt er verslag gedaan van het ontwikkelen van het prototype.

\end{abstract}
\begin{todolist}
  \item[\done] Inleiding en introductie
  \item[\done] Critical Review
  \item[\done] Methode
  \item Uitvoering en resultaten
  \item Conclusie en reflectie
\end{todolist}
