% \section*{Abstract}
\begin{abstract}
Dit Supportive Narrative gaat over technologie in het distributieproces van muziek. Streaming is de afgelopen jaren een belangrijke vorm van muziekconsumptie geworden. Het lijkt echter dat de grote voordelen die artiesten van streaming zouden hebben alleen gelden bij artiesten die al bekend en rendabel zijn. Een oplossing die in dit Supportive Narrative wordt beschreven is het teruggaan naar fysieke verkoop van muziek. Er zijn verschillende argumenten die dit ondersteunen, welke worden besproken en onderbouwd in dit Supportive Narrative. Verder wordt er een prototype-streamingdienst ontwikkeld die de voordelen van streaming combineert met de voordelen van fysieke verkoop. Het prototype maakt gebruik van NFC om een fysiek artefact te koppelen aan een digitaal product.

In de critical review wordt gekeken naar verschillende geluid-dragende media en hoe zij invloed hebben gehad op de muziekindustrie. Hierbij wordt beschreven wat de disruptieve werking is van het betreffende medium ten opzichte van zijn voorganger en wat het probleem is dat het medium heeft proberen op te lossen. Deze informatie wordt later gebruikt om context te bieden aan het businessplan en het prototype.

Er zijn twee artefacten: een prototype streamingdienst en een uitgeschreven businessplan. Deze twee artefacten bouwen op elkaar voort, en zijn synchroon opgezet en ontwikkeld. Het werken aan  het businessplan bestaat uit het onderbouwen van verschillende aannames die daarmee het businessplan en het prototype verder ondersteunen. Het prototype is een implementatie van het businessplan, en is een werkend product. Bij het ontwikkelen van het prototype zijn voor mij nieuwe ontwikkelmethoden gebruikt, zoals een combinatie van lineair en iteratief werken, en een manier om uitstelgedrag functioneel in te zetten.

In het hoofdstuk uitvoering en resultaten wordt het businessplan verder uiteengezet. Dit wordt gedaan door middel van het onderbouwen van de aannames en het beschrijven van de verschillende sub-producten. Verder wordt er verslag gedaan van het ontwikkelen van het prototype waarbij verschillende ontwerpkeuzes worden uitgelicht. 

Tot slot wordt er in het hoofdstuk conclusie gereflecteerd op het proces en project. Inhoudelijk blijken er een paar zaken te zijn waar ik meer onderzoek naar had kunnen doen, zoals de rechten van muziek, en de manier waarop luisteraars per stream zouden betalen. Verder wordt er gereflecteerd op de status van het project en businessplan, en worden toekomstplannen besproken. Ook wordt er gereflecteerd op het proces, en wordt gekeken of de nieuwe ontwikkelmethoden hebben gewerkt.

\end{abstract}
