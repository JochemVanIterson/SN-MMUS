\section{Introductie}
% In dit hoofdstuk wordt het onderwerp geïntroduceerd, dat gerelateerd is aan de eigen praktijk van de student. Een aantal punten zijn daarbij van belang:
% -Het doel van de paper of thesis: wat gaat het stuk uiteindelijk opleveren?
% -Het resultaat: wat is er af als het af is?
% -Het onderwerp: wat is het verschijnsel, de technologie, het artefact, de persoon, enzovoort, waarover de paper gaat?
% -De context van het onderwerp: wat is het grotere kader?
% -Relevantie: waarom is het onderwerp van belang voor anderen?
% -Motivatie: waarom is het onderwerp relevant voor de eigen praktijk?

% ---------------------------------------------------------------------------------------- %
\subsection{Doel}
% Het doel van de paper of thesis: wat gaat het stuk uiteindelijk opleveren?
Het doel van dit SN is om een contextueel onderzoek te doen bij een bestaand, lang lopend project. Door dit onderzoek krijgt het concept van dit project een beter beargumenteerde basis, waarmee in een later stadium naar investeerders gestapt kan worden. Door (markt) onderzoek te doen naar vernieuwende technologieën kan ik in de toekomst de concepten van mijn projecten beter onderbouwen. Momenteel is veel van mijn argumentatie gebaseerd op persoonlijke ervaring en dat van anderen.

% ---------------------------------------------------------------------------------------- %
\subsection{Verwachtte resultaat}
% Het resultaat: wat is er af als het af is?
Ik verwacht twee artefacten te genereren uit dit SN.

\subsubsection*{Businessmodel}
Het eerste en belangrijkste artefact uit dit SN is een uitgewerkt en onderbouwd businessmodel met uitgebreide argumentatie. Hoewel het concept van de NichePlayer al enige tijd bestaat en er al een aantal werkende prototypes zijn ontwikkeld, zijn de aannames voor het project vooral gemaakt op persoonlijke ervaring. 

\begin{itemize}
    \item Uitgeschreven en vooral correct onderbouwd businessmodel
    \item Eerste stappen van realiseren businessmodel reeds uitgevoerd
\end{itemize}

\subsubsection*{Prototype}
Een tweede artefact is een werkend prototype van het systeem. Omdat ik een maker ben is het voor mij belangrijk te blijven ontwikkelen en maken tijdens het schrijven van een SN of Thesis. Het schrijven en ontwikkelen maakt een wisselwerking, waarmee ik ideeën uitschrijf in mijn SN, prototype in code, en vervolgens weer reflecteer in het SN. Beide moeten voor mij bestaan.

Het is ook relevant voor het onderzoek. Door te blijven ontwikkelen kan ik concepten van klanten en investeerders vrijwel direct laten zien, en ze op deze manier betrekken in het (ontwerp) proces.

\begin{itemize}
    \item Werkend prototype van het systeem
    \item Eerste live use cases in productie
\end{itemize}

% ---------------------------------------------------------------------------------------- %
\subsection{Onderwerp}
% Het onderwerp: wat is het verschijnsel, de technologie, het artefact, de persoon, enzovoort, waarover de paper gaat?
\begin{itemize}
    \item NichePlayer
\end{itemize}
De NichePlayer is een open-source en self-hostable streaming service. 

% ---------------------------------------------------------------------------------------- %
\subsection{Context}
% De context van het onderwerp: wat is het grotere kader?
Technologie in het distributieproces van muziek. Het evenwicht tussen het gemak van digitaal en de waarde van het fysieke artefact.

% ---------------------------------------------------------------------------------------- %
\subsection{Relevantie}
% Relevantie: waarom is het onderwerp van belang voor anderen?

% ---------------------------------------------------------------------------------------- %
\subsection{Motivatie}
% Motivatie: waarom is het onderwerp relevant voor de eigen praktijk?
\begin{itemize}
    \item Artiesten helpen
\end{itemize}
