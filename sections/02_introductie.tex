\section{Introductie}
% In dit hoofdstuk wordt het onderwerp geïntroduceerd, dat gerelateerd is aan de eigen praktijk van de student. Een aantal punten zijn daarbij van belang:
% -Het doel van de paper of thesis: wat gaat het stuk uiteindelijk opleveren?
% -Het resultaat: wat is er af als het af is?
% -Het onderwerp: wat is het verschijnsel, de technologie, het artefact, de persoon, enzovoort, waarover de paper gaat?
% -De context van het onderwerp: wat is het grotere kader?
% -Relevantie: waarom is het onderwerp van belang voor anderen?
% -Motivatie: waarom is het onderwerp relevant voor de eigen praktijk?

% ---------------------------------------------------------------------------------------- %
% Context
De distributie van muzikale content is een gebied waar veel in wordt geïnnoveerd. Waar in de middeleeuwen distributie van muziek letterlijk een monnikenwerk was, is nu vrijwel alle muziek toegankelijk met een enkele klik op je telefoon. Met de komst van computers en (snel) internet is de weg vrijgemaakt voor streamingdiensten met een haast oneindige database aan muziek.

Deze onbeperkte database aan muziek is niet alleen prettig voor de luisteraar, ook de artiest heeft een groot voordeel aan deze toegang: het is makkelijker om een groot publiek te bereiken. Muziek op streamingdiensten kost minder productiekosten (geen CD druk) en is makkelijker te verspreiden over de hele wereld.

% Probleemstelling
Een nieuw probleem is echter dat de grote voordelen van streaming diensten vooral werkt voor artiesten die al bekend zijn. Kleine en beginnende artiesten moeten veel streams genereren om te kunnen leven van hun muziek verkoop, wat in het begin van hun carrière nog lastig kan zijn. Kleine artiesten hebben baat bij een hechte fanbase, wat moeilijk te bereiken is op de grote streamingdiensten van tegenwoordig.

% Oplossingsrichting
Een oplossing is focussen op inkomsten uit optredens. 

Een andere richting voelt als een stap terug in de tijd: fysieke verkoop. Sinds enkele jaren is de verkoop van vinyl platen weer gestegen \todo{Bron zoeken}. Deze platen worden vaak niet gebruikt om muziek van te luisteren, de kwaliteit is veel slechter dan wat we gewend zijn van streamingdiensten. Waar deze platen wel voor werken is als aandenken of pronkstuk in de kast. 

% ---------------------------------------------------------------------------------------- %
\subsection{Onderwerp}
% Het onderwerp: wat is het verschijnsel, de technologie, het artefact, de persoon, enzovoort, waarover de paper gaat?
Binnen dit Supportive Narrative zal onderzoek gedaan gaan worden naar fysieke verkoop van content binnen de steeds meer digitale muziekindustrie. Door de digitalisering van de afgelopen 20 jaar is de verkoop van fysieke media sterk gedaald \cite{dong2022valueofmusic}. Dit komt door het gemak en de toegankelijkheid die digitale platformen hebben gemaakt. In plaats van een kast vol met CD's en LP's heeft een gebruiker van een streamingdienst toegang tot miljoenen nummers op een server. Het probleem hiervan is dat het individuele nummer hierdoor bijna geen waarde meer lijkt te hebben.

Fysieke producten aan de andere kant hebben veel eigen waarde. Ze nemen fysieke ruimte in beslag, hebben productiekosten, een beperkte oplage, en beschadigen over de tijd. Nog belangrijker is dat fysieke producten emotionele waarde kunnen hebben; de gebruiker zal zich bijvoorbeeld herinneren dat een CD op een bepaald festival is gekocht, het kan een bepaalde beschadiging hebben vanwege overmatig gebruik, of er staat een handtekening van de artiest geschreven op de album hoes van een LP.

Het is moeilijk om deze eigenschappen direct over te nemen bij digitale producten. Digitaal is nooit uniek. Waar in het geval van reproductie via fysieke media vrijwel altijd de kwaliteit per reproductie omlaag gaat is het in computers mogelijk een perfecte kopie te maken van het origineel. Daarnaast is computer opslag de laatste jaren zo groot dat er praktisch geen limiet zit op hoeveel muziek van perfecte kwaliteit lokaal kan worden opgeslagen. Tot slot hebben recente de innovaties in internet het mogelijk gemaakt hoge kwaliteit muziek te kunnen streamen zonder merkbare buffering of vertraging.

Het onderwerp van dit SN gaan over het vinden van een evenwicht tussen het fysieke artefact en het gemak en kwaliteit van digitale toegang. De keuzes en onderbouwingen zullen worden beschreven in het SN en worden gebruikt om het businessplan en het prototype van de NichePlayer te ondersteunen.

\subsection{Doel}
% Het doel van de paper of thesis: wat gaat het stuk uiteindelijk opleveren?
Het doel van dit SN is om een contextueel onderzoek te doen bij een bestaand, lang lopend project: de NichePlayer. Via dit onderzoek krijgt het concept van dit project een goede beargumenteerde basis, waarmee in een later stadium van ontwikkeling naar investeerders gegaan kan worden. Door tijdens het SN onderzoek te doen naar vernieuwende technologieën kan ik bij mijn toekomstige projecten de concepten beter onderbouwen. Momenteel is de argumentatie van mijn projecten grotendeels gebaseerd op persoonlijke ervaring en de verhalen van anderen.

Enkele aannames die door middel van dit SN gecontroleerd moeten worden zijn de volgende:

\begin{itemize}
    \item Door het vernieuwde luistergedrag binnen streamingdiensten heeft het album als distributievorm zijn individuele waarde verloren
    \item De waarde zit op dit moment niet meer in de muziek, maar het platform die toegang biedt tot de (haast onbeperkte) content
    \item De waarde van muziek zit in de community die wordt gevormd rondom de muziek
    \item Doordat beginnende artiesten op streamingdiensten direct concurreren met artiesten op professioneel niveau is het moeilijk om door te breken. Dit is slecht voor de diversiteit binnen de muziekindustrie
    \item Hoewel het steeds makkelijker is om muziek te uploaden, zorgt dit er ook voor dat de muziek kwijt raakt in het grote geheel
\end{itemize}

De bovengenoemde aannames gaan voornamelijk over kleine en beginnende artiesten en kunstenaars. Naast dat dit een relatief makkelijke doelgroep is om te bereiken is dit ook de groep die het meeste bereidt zou zijn om te veranderen. Een artiest op professioneel niveau heeft vaak al een platform, fanbase en workflow en zal hier minder snel van afwijken om iets nieuws te proberen.

% ---------------------------------------------------------------------------------------- %
\subsection{Resultaten}
% Het resultaat: wat is er af als het af is?
Uit dit SN moeten twee artefacten gaan komen: een businessplan en een prototype van het product.

\subsubsection*{Businessplan}
Het eerste en belangrijkste artefact uit dit SN is een uitgewerkt businessplan met onderbouwing en argumentatie. Hoewel ik al lange tijd bezig ben met het concept van de NichePlayer en er al een aantal werkende prototypes zijn ontwikkeld, zijn de aannames voor het project vooral gemaakt op persoonlijke ervaring. Door te gaan werken aan het businessplan en deze te onderbouwen met onderzoeken uit dit SN wordt een basis gelegd voor het project en mijn latere beroepspraktijk.

Dit artefact bestaan op zijn minst uit een uitgeschreven businessplan met argumentatie en voorbeelden. Daarnaast is het de bedoeling dat de eerste stappen van het realiseren van het businessplan zijn uitgevoerd.

\subsubsection*{Prototype}
Een tweede artefact is een werkend prototype van het systeem. Omdat ik een maker ben is het voor mij belangrijk te blijven ontwikkelen en prototypen tijdens het schrijven van het SN. Daarnaast is dit ook de basis van Practice Based Research. Het schrijven en ontwikkelen maakt een wisselwerking, waarmee ik ideeën uitwerk in mijn SN, prototype in code, en vervolgens weer kan reflecteren in het SN. Beide moeten voor mij bestaan.

Het synchroon ontwikkelen van een prototype is ook relevant voor het onderzoek. Door te blijven ontwikkelen kan ik concepten van klanten en investeerders vrijwel direct laten zien, en ze op deze manier betrekken in het (ontwerp) proces. Deze manier van werken pas ik al toe in mijn beroepspraktijk.

\subsubsection*{MoSCoW}
Om voor mijzelf een overzicht te maken wat binnen dit SN moet gebeuren heb ik een MoSCoW opgezet. De MoSCoW methode wordt vaak binnen de softwareontwikkeling gebruikt om de prioriteit van taken te beschrijven. Deze prioriteiten zijn verdeeld in vier groepen: \textit{Must have} is een vereiste van het project. Als een must-have niet is verwerkt in het eindresultaat kan het project worden gezien als gefaald. \textit{Should have} zijn punten die zeer interessant zijn voor het project, maar niet noodzakelijk voor de kern van het project en dus ook om het project te laten slagen. Punten binnen de \textit{Could have} zijn interessante toevoegingen voor bijvoorbeeld de gebruikerservaring, en kosten over het algemeen weinig tijd om toe te voegen. Vaak worden deze punten toegevoegd wanneer aan het einde nog tijd over is. \textit{Won't have} zijn ideeën voor in de toekomst. Deze punten zitten vanwege tijdgebrek vrijwel nooit in het eindproduct, en hebben vaak de minste impact in de werking van het eindresultaat van (deze iteratie) van het project.

\parbox{\textwidth}{
    \textbf{Must have}
    \begin{itemize}
        \item Argumentatie die het concept ondersteunt
        \item Business Model Canvas
        \item Uitgewerkt en uitgeschreven businessplan
    \end{itemize}
    \textbf{Should have}
    \begin{itemize}
        \item Eerste stappen van realiseren businessplan uitgevoerd
        \item Werkend prototype voor intern gebruik
    \end{itemize}
    \textbf{Could have}
    \begin{itemize}
        \item Prototype in gebruik bij een testpubliek
        \item Eerste inkomsten
    \end{itemize}
    \textbf{Won't have}
    \begin{itemize}
        \item Volledig draaiend bedrijf
    \end{itemize}
}

% ---------------------------------------------------------------------------------------- %
\subsection{Relevantie}
% Relevantie: waarom is het onderwerp van belang voor anderen?
De concepten en de dienst die in dit SN worden beschreven zijn relevant binnen de distributieprocessen van de muziek industrie. Het biedt beginnende en kleine artiesten, producers en kunstenaars de kans een grotere inkomsten stroom te genereren uit de verkoop van hun muziek. Dit zou momenteel niet mogelijk zijn met bestaande diensten als Spotify en Bandcamp.

% ---------------------------------------------------------------------------------------- %
\subsection{Motivatie}
% Motivatie: waarom is het onderwerp relevant voor de eigen praktijk?
% Artiesten helpen
Als ontwikkelaar van creatieve soft- en hardware maak ik voornamelijk toepassingen voor anderen. Via mijn praktijk wil ik artiesten helpen in hun creatieve proces.

% Mensen verbinden en communities vormen
Daarnaast wil ik met mijn producten mensen met elkaar verbinden en communities vormen. 

% Grote groep kennissen is artiest
Een groot deel van mijn kennissen is artiest. Dit zorgt er voor dat ik bekend ben met de zorgen en problemen die ze hebben omtrent het delen van hun muziek. Ook betekend dit dat ik makkelijk toegang heb tot mijn doelgroep en daarmee een testpubliek heb voor mijn project.

% Business opzetten
Het businessplan wat wordt uitgewerkt in dit SN zal na mijn afstuderen een opstap zijn om mijn eigen bedrijf en beroepspraktijk vorm te geven. Na mijn afstuderen moet ik een inkomstenstroom gaan genereren. Mijn voorkeur gaat hierbij uit naar inkomsten uit eigen projecten, hoewel ik de eerste periode ergens anders in loondienst zal moeten om tijd te overbruggen.

Naast dat het onderwerp zoals hierboven genoemd relevant is voor mij is ook de vorm waarop onderzocht gaat worden relevant. Ik werk veel op aannames en onderbuikgevoel. Door (geforceerd) een businessplan uit te werken wordt ik gedwongen eerst na te denken en te beargumenteren wat ik wil voordat ik aan de slag ga met ontwikkelen.

\subsection{Outline}
% Briefly talk about the rest of the SN
