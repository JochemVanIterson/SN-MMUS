\section{Introductie}
% In dit hoofdstuk wordt het onderwerp geïntroduceerd, dat gerelateerd is aan de eigen praktijk van de student. Een aantal punten zijn daarbij van belang:
% -Het doel van de paper of thesis: wat gaat het stuk uiteindelijk opleveren?
% -Het resultaat: wat is er af als het af is?
% -Het onderwerp: wat is het verschijnsel, de technologie, het artefact, de persoon, enzovoort, waarover de paper gaat?
% -De context van het onderwerp: wat is het grotere kader?
% -Relevantie: waarom is het onderwerp van belang voor anderen?
% -Motivatie: waarom is het onderwerp relevant voor de eigen praktijk?

% ---------------------------------------------------------------------------------------- %
% Context
De distributie van muzikale content is een gebied waar veel in wordt geïnnoveerd. Waar in de middeleeuwen distributie van muziek letterlijk een monnikenwerk was, is nu vrijwel alle muziek toegankelijk met een enkele klik op je telefoon. Met de komst van computers en (snel) internet is de weg vrijgemaakt voor streamingdiensten met een haast oneindige database aan muziek.

Deze onbeperkte database aan muziek is niet alleen prettig voor de luisteraar, ook de artiest heeft een aantal grote voordelen aan deze toegankelijkheid van zijn muziek. Zo is het makkelijker om een groot publiek te bereiken. Internet en streaming is niet gebonden aan bepaalde plekken of tijden, en is dus altijd beschikbaar.

Daarnaast is het ook makkelijker om muziek uit te brengen en te verspreiden. Muziek op streamingdiensten kost minder productiekosten omdat er geen fysieke media als de CD of LP gedrukt hoeft te worden, en het hoeft ook niet geleverd te worden aan winkels.

% Probleemstelling
Een nieuw probleem is echter dat de grote voordelen van streaming diensten vooral lijken te werken voor artiesten die al bekend zijn. Kleine en beginnende artiesten moeten veel streams genereren om te kunnen leven van hun muziek verkoop, wat in het begin van hun carrière nog lastig kan zijn. Om hun muziek bekend te maken sluiten veel artiesten aan bij labels, en schrijven ze veel naar de makers van playlists. Deze twee partijen forceren de artiest vaak om muziek te maken die past in de huidige trends, wat dan ten koste kan gaan van de creativiteit en diversiteit van de muziek.

% Oplossingsrichting optreden
Een oplossing is meer focussen op inkomsten uit liveoptredens. Door het houden van optredens vormt een artiest sneller een hechte fanbase en voelt de muziek veel persoonlijker voor de luisteraar. Een probleem van optreden is echter wel dat het een vorm is van actief inkomen. Voor het houden van een optreden moet een artiest naar de plek toe reizen met alle apparatuur, het optreden houden, en dan weer terugreizen. Dit kost de artiest erg veel tijd en energie ten opzichte van een passief inkomen zoals verkoop van de muziek. Bij verkoop is de artiest vaak niet zelf betrokken en wordt het werk gedaan door derden. Een ander probleem van optreden is dat er festivals en concerten nodig zijn om op te treden. Door de coronapandemie zijn deze evenementen de afgelopen jaren niet mogelijk geweest, waardoor veel artiesten ander werk zijn gaan zoeken om toch aan inkomsten te komen.

% Oplossingsrichting fysieke verkoop
Een andere richting dan optreden voelt misschien snel als een stap terug in de tijd: fysieke verkoop. Na jaren van verlies is de verkoop van vinyl platen dit jaar voor het eerst weer gestegen \cite{year_end_2022_RIAA_revenue_statistics}. Deze platen worden vaak niet gebruikt om muziek van te luisteren. De kwaliteit is veel slechter dan wat we tegenwoordig gewend zijn van streamingdiensten en de meeste mensen hebben geen platenspeler meer in huis. Waar deze platen wel voor worden gekocht is als verzamelitem, aandenken of pronkstuk in de kast. De rol van het medium is hiermee veranderd van een functioneel opslagmedium naar een fysiek artefact met veel emotionele waarde. Fysieke verkoop van platen en merchandise wordt door luisteraars en fans gebruikt om zich te identificeren. Het is een manier om te laten zien dat je fan bent van een artiest, en om je te onderscheiden van anderen.

% ---------------------------------------------------------------------------------------- %
\subsection{Onderwerp}
% Het onderwerp: wat is het verschijnsel, de technologie, het artefact, de persoon, enzovoort, waarover de paper gaat?
Binnen dit Supportive Narrative zal onderzoek gedaan gaan worden naar fysieke verkoop van content binnen de steeds meer digitale muziekindustrie. Door de digitalisering van de afgelopen 20 jaar is de verkoop van fysieke media sterk gedaald \cite{dong2022valueofmusic}. Dit komt door het gemak en de toegankelijkheid die digitale platformen hebben gebracht. In plaats van een kast vol met CD's en LP's heeft een gebruiker van een streamingdienst toegang tot miljoenen nummers op een server. De nummers staan daarnaast ook niet vast. Wanneer een nieuw nummer wordt uitgebracht is deze direct beschikbaar voor de gebruiker. Het probleem van deze onbeperkte toegang is echter dat het individuele nummer hierdoor bijna geen waarde meer lijkt te hebben.

Fysieke producten aan de andere kant hebben veel waarde door hun fysieke eigenschappen. Dit komt omdat ze ruimte in beslag nemen, geld kosten om te produceren, een beperkte oplage hebben, en beschadigen over de tijd. Nog belangrijker is dat fysieke producten emotionele waarde kunnen hebben; de gebruiker zal zich bijvoorbeeld herinneren dat een CD op een bepaald festival is gekocht, het kan een bepaalde beschadiging hebben vanwege overmatig gebruik, of er staat een handtekening van de artiest.

Het is moeilijk om deze eigenschappen direct over te nemen bij digitale producten. Digitale media is nooit uniek. Waar in het geval van reproductie via fysieke media vrijwel altijd de kwaliteit per reproductie omlaag gaat is het in computers mogelijk een perfecte kopie te maken van het origineel. Daarnaast is computer opslag de laatste jaren zo groot dat er praktisch geen limiet zit op hoeveel muziek (van perfecte kwaliteit) lokaal kan worden opgeslagen. Tot slot hebben recente innovaties in internet het mogelijk gemaakt hoge kwaliteit muziek te kunnen streamen zonder merkbare buffering of vertraging.

Het onderwerp van dit SN gaan over het vinden van een evenwicht tussen het fysieke artefact en het gemak en kwaliteit van digitale toegang. De keuzes en onderbouwingen zullen worden beschreven in het SN en worden gebruikt om het businessplan en het prototype van de NichePlayer te ondersteunen.

\subsection{Doel}
% Het doel van de paper of thesis: wat gaat het stuk uiteindelijk opleveren?
Het doel van dit SN is om een contextueel onderzoek te doen bij een bestaand, lang lopend project: de NichePlayer. De NichePlayer is een streamingdienst die speciaal wordt ontworpen met creativiteit, openheid en transparantie in gedachte. Door de NichePlayer te gebruiken krijgt de artiest controle terug over het medium in de vorm van inkomsten en layout. De gebruiker krijgt de mogelijkheid om de artiest te steunen door middel van fysieke verkoop gekoppeld aan toegang tot de muziek. 

Via het onderzoek van dit Supportive narrative krijgt het concept van dit project een goede beargumenteerde basis, waarmee in een later stadium van ontwikkeling naar investeerders gegaan kan worden. Momenteel is de argumentatie van mijn projecten grotendeels gebaseerd op persoonlijke ervaring en de verhalen van anderen. Dit was goed voor de prototype fase, maar om verder te komen is een betere onderbouwing nodig.

Enkele aannames die door middel van dit SN gecontroleerd moeten worden zijn de volgende:

\begin{itemize}
    \item Door het vernieuwde luistergedrag binnen streamingdiensten heeft het album als distributievorm zijn individuele waarde verloren.
    \item De waarde zit op dit moment niet meer in de muziek, maar in het platform dat toegang biedt tot de voor de luisteraar haast onbeperkte hoeveelheid aan content.
    \item De waarde van muziek zit in de community die wordt gevormd rondom de muziek.
    \item Doordat beginnende artiesten op streamingdiensten direct concurreren met artiesten op professioneel niveau is het moeilijk om door te breken. Dit is slecht voor de diversiteit binnen de muziekindustrie.
    \item Hoewel het steeds makkelijker is om muziek te uploaden, zorgt dit er ook voor dat de muziek kwijtraakt in het grote geheel.
\end{itemize}

De bovengenoemde aannames gaan voornamelijk over kleine en beginnende artiesten en kunstenaars. Naast dat dit een relatief makkelijke doelgroep is om te bereiken is dit ook de groep die het meeste bereidt zal zijn om te veranderen. Een artiest op professioneel niveau heeft vaak al een platform, fanbase en workflow en zal hier minder snel van afwijken om iets nieuws te proberen. Artiesten die zelfs daarboven zitten en hun pensioen bij wijze van al binnen hebben zouden wel open kunnen staan voor experimenten, maar staan te ver buiten mijn netwerk om te bereiken.

% ---------------------------------------------------------------------------------------- %
\subsection{Resultaten}
% Het resultaat: wat is er af als het af is?
Uit dit SN moeten twee artefacten gaan komen: een businessplan en een prototype van het product.

\subsubsection*{Businessplan}
Het eerste en belangrijkste artefact uit dit SN is een uitgewerkt businessplan met onderbouwing en argumentatie. Hoewel ik al lange tijd bezig ben met het concept van de NichePlayer en er al een aantal werkende prototypes zijn ontwikkeld, zijn de aannames voor het project vooral gemaakt op persoonlijke ervaring. Door te gaan werken aan het businessplan en deze te onderbouwen met onderzoeken uit dit SN wordt een basis gelegd voor het project en mijn latere beroepspraktijk.

Dit artefact bestaat op zijn minst uit een uitgeschreven businessplan met argumentatie en voorbeelden. Daarnaast is het de bedoeling dat de eerste stappen van het realiseren van het businessplan zijn uitgevoerd.

\subsubsection*{Prototype systeem}
Een tweede artefact is een werkend prototype van het systeem. Omdat ik een maker ben is het voor mij belangrijk te blijven ontwikkelen en prototypen tijdens het schrijven van het SN. Daarnaast is dit ook de basis van Practice Based Research. Het schrijven en ontwikkelen geeft een wisselwerking waarmee ik ideeën uitwerk in mijn SN, prototype in code, en vervolgens weer kan reflecteren in het SN. Beide moeten voor mij bestaan.

Het synchroon ontwikkelen van een prototype is ook relevant voor het onderzoek. Door te blijven ontwikkelen kan ik concepten van klanten en investeerders vrijwel direct laten zien, en ze op deze manier betrekken in het (ontwerp) proces. Deze manier van werken pas ik al toe in mijn beroepspraktijk.

Het prototype van de NichePlayer moet een werkend systeem zijn voor intern gebruik en demo doeleinden. Het MVP (Minimum Viable Product) moet een systeem zijn waarbij gebruikers muziek kunnen afspelen wat achter een toegang-beveiliging zit. Gebruikers moeten zich hierbij kunnen registreren en inloggen in het systeem. De muziek moet vervolgens kunnen worden afgespeeld zoals dat ook gaat in bestaande diensten met functies als pauzeren, afspeelpositie wijzigen, vorige en volgende track in een album, etc. Daarnaast moet het afspeelgedrag kunnen worden opgeslagen om zo een compleet overzicht te kunnen geven aan de artiest van het luistergedrag van de gebruikers.

\subsubsection*{MoSCoW}
Om voor mijzelf een overzicht te maken wat binnen dit SN moet gebeuren heb ik een MoSCoW opgezet. De MoSCoW methode wordt vaak binnen de softwareontwikkeling gebruikt om de prioriteit van taken te beschrijven. Deze prioriteiten zijn verdeeld in vier groepen: \textit{Must have}, \textit{Should have}, \textit{Could have} en \textit{Won't have}. \textit{Must have} is een vereiste van het project. Als een must-have niet is verwerkt in het eindresultaat kan het project worden gezien als gefaald. \textit{Should have} zijn punten die zeer interessant zijn voor het project, maar niet noodzakelijk voor de kern en dus ook niet noodzakelijk om het project te laten slagen. Punten binnen de \textit{Could have} zijn interessante toevoegingen voor het project, bijvoorbeeld de gebruikerservaring. Vaak kosten deze punten weinig tijd om toe te voegen, en worden ze uitgevoerd wanneer aan het einde nog tijd over is. \textit{Won't have} zijn ideeën voor in de toekomst. Deze punten zitten vanwege tijdgebrek vrijwel nooit in het eindproduct, en hebben vaak de minste impact op de werking van het eindresultaat van (deze iteratie) van het project.

\parbox{\textwidth}{
    \textbf{Must have}
    \begin{itemize}
        \item Argumentatie die het concept ondersteunt
        \item Business Model Canvas
        \item Uitgewerkt en uitgeschreven businessplan
    \end{itemize}
    \textbf{Should have}
    \begin{itemize}
        \item Eerste stappen van realiseren businessplan uitgevoerd
        \item Werkend prototype voor intern gebruik
    \end{itemize}
    \textbf{Could have}
    \begin{itemize}
        \item Prototype in gebruik bij een testpubliek
        \item Eerste inkomsten
    \end{itemize}
    \textbf{Won't have}
    \begin{itemize}
        \item Volledig draaiend bedrijf
    \end{itemize}
}

% ---------------------------------------------------------------------------------------- %
\subsection{Relevantie}
% Relevantie: waarom is het onderwerp van belang voor anderen?
De concepten en de dienst die in dit SN worden beschreven zijn relevant binnen de distributieprocessen van de muziekindustrie. Het biedt beginnende en kleine artiesten, producers en kunstenaars de kans een grotere inkomstenstroom te genereren uit de verkoop van hun muziek. Dit is momenteel niet mogelijk met bestaande diensten als Spotify en Bandcamp. Daarnaast is een oplossing vinden voor het geschetste probleem goed voor de diversiteit van de muziekindustrie. Artiesten schrijven nu veel muziek die 'werkt' binnen de muziekindustrie en waar ze snel veel streams mee kunnen krijgen.

Dit onderzoek en het project bieden een service zonder dat de artiest zich hoeft aan te passen aan de industrie. Daarnaast bepaald de artiest zelf hoe de muziek wordt aangeboden en hoeveel het kost. Dit geeft de artiest de macht terug over zijn eigen muziek.

% ---------------------------------------------------------------------------------------- %
\subsection{Motivatie}
% Motivatie: waarom is het onderwerp relevant voor de eigen praktijk?
% Artiesten helpen
Als ontwikkelaar van creatieve soft- en hardware maak ik voornamelijk toepassingen voor anderen. Via mijn praktijk wil ik artiesten helpen in hun creatieve proces. Dit doe ik door het ontwikkelen van tools die hun werk makkelijker maken. Omdat ik zelf ook in bands heb gespeeld kan ik goed levelen met artiesten en kunstenaars. Ik snap hoe ze denken en wat ze nodig hebben in hun proces en werk. Daarnaast ben ik als live-technicus actief in het hoge segment van de Nederlandse muziekindustrie. Dit geeft mij een uniek perspectief op de muziekindustrie.

% Mensen verbinden en communities vormen
Met mijn producten probeer ik mensen met elkaar te verbinden.

% Grote groep netwerk is artiest
Een groot deel van mijn netwerk is artiest. Dit zorgt ervoor dat ik bekend ben met de zorgen en problemen die ze hebben rond het delen en publiceren van hun muziek. Ook betekent dit dat ik makkelijk toegang heb tot mijn doelgroep en daarmee een testpubliek heb voor mijn project.

% Business opzetten
Het businessplan wat wordt uitgewerkt in dit SN zal na mijn afstuderen een opstap zijn om mijn eigen bedrijf en beroepspraktijk vorm te geven. Na mijn afstuderen moet ik een inkomstenstroom gaan genereren. Een groot deel van mijn inkomsten zal uit eigen projecten moeten komen, met als uiteindelijke voorkeur 60/40 loondienst/eigen projecten. Deze verdeling is gebaseerd op mijn ervaring uit mijn stage. Het is fijn om een vast inkomen te hebben, maar wanneer ik te lang aan eenzelfde project werk raak ik mijn creativiteit en motivatie kwijt. Daarnaast is het voor mij belangrijk om mijn eigen projecten te kunnen blijven ontwikkelen omdat daar nieuwe kennis en ervaring uit voort komt.

Naast dat het onderwerp zoals hierboven genoemd relevant is voor mij is ook de vorm waarop onderzocht gaat worden relevant. Ik werk veel op aannames en onderbuikgevoel. Door (geforceerd) een businessplan uit te werken word ik gedwongen eerst na te denken en te beargumenteren wat ik wil voordat ik aan de slag ga met ontwikkelen.

% \subsection{Outline}
% Briefly talk about the rest of the SN
