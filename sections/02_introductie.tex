\section{Introductie}
% In dit hoofdstuk wordt het onderwerp geïntroduceerd, dat gerelateerd is aan de eigen praktijk van de student. Een aantal punten zijn daarbij van belang:
% -Het doel van de paper of thesis: wat gaat het stuk uiteindelijk opleveren?
% -Het resultaat: wat is er af als het af is?
% -Het onderwerp: wat is het verschijnsel, de technologie, het artefact, de persoon, enzovoort, waarover de paper gaat?
% -De context van het onderwerp: wat is het grotere kader?
% -Relevantie: waarom is het onderwerp van belang voor anderen?
% -Motivatie: waarom is het onderwerp relevant voor de eigen praktijk?

% ---------------------------------------------------------------------------------------- %
\subsection{Doel}
% Het doel van de paper of thesis: wat gaat het stuk uiteindelijk opleveren?
Het doel van dit SN is om een contextueel onderzoek te doen bij een bestaand, lang lopend project. Door dit onderzoek krijgt het concept van dit project een goede beargumenteerde basis, waarmee in een later stadium van ontwikkeling naar investeerders gegaan kan worden. Door tijdens het SN onderzoek te doen naar vernieuwende technologieën kan ik bij toekomstige projecten de concepten van beter onderbouwen. Momenteel is de argumentatie van mijn projecten grotendeels gebaseerd op persoonlijke ervaring en dat van anderen.

Enkele aannames die door middel van dit SN gecontroleerd moeten worden zijn de volgende:

\begin{itemize}
    \item Door het vernieuwde luistergedrag binnen streamingdiensten heeft het album als distributievorm zijn individuele waarde verloren
    \item De waarde zit op dit moment niet meer in de muziek, maar het platform die toegang biedt tot een haast onbeperkte hoeveelheid content
    \item Doordat beginnende artiesten op streamingdiensten direct concurreren met artiesten op professioneel niveau is het moeilijk om door te breken. Dit is slecht voor de diversiteit binnen de muziekindustrie.
    \item Hoewel het steeds makkelijker is om muziek te uploaden, zorgt dit er ook voor dat de muziek kwijt raakt in het grote geheel
\end{itemize}
% ---------------------------------------------------------------------------------------- %
\subsection{Resultaten}
% Het resultaat: wat is er af als het af is?
Uit dit SN moeten twee artefacten gaan komen; een businessplan en een bijbehorend prototype.

\subsubsection*{Businessplan}
Het eerste en belangrijkste artefact uit dit SN is een uitgewerkt businessplan met uitgebreide onderbouwing en argumentatie. Hoewel ik al lange tijd bezig ben met het concept van de NichePlayer, en er al een aantal werkende prototypes zijn ontwikkeld, zijn de aannames voor het project vooral gemaakt op persoonlijke ervaring. Door te gaan werken aan het business en deze te onderbouwen met onderzoeken uit dit SN wordt een basis gelegd voor mijn latere beroepspraktijk.

Dit artefact bestaan op zijn minst uit een uitgeschreven business met argumentatie en voorbeelden. Daarnaast is het de bedoeling dat de eerste stappen van het realiseren van het business zijn uitgevoerd.

\subsubsection*{Prototype}
Een tweede artefact is een werkend prototype van het systeem. Omdat ik een maker ben is het voor mij belangrijk te blijven ontwikkelen en maken tijdens het schrijven van een SN of Thesis. Daarnaast is dit ook de basis van Practice Based Research. Het schrijven en ontwikkelen maakt een wisselwerking, waarmee ik ideeën uitwerk in mijn SN, prototype in code, en vervolgens weer kan reflecteren in het SN. Beide moeten voor mij bestaan.

Het synchroon ontwikkelen van een prototype is ook relevant voor het onderzoek. Door te blijven ontwikkelen kan ik concepten van klanten en investeerders vrijwel direct laten zien, en ze op deze manier betrekken in het (ontwerp) proces.

\subsubsection*{MoSCoW}
Om voor mijzelf een overzicht te maken wat binnen dit SN moet gebeuren heb ik een MoSCoW opgezet. De MoSCoW methode wordt vaak binnen de softwareontwikkeling gebruikt om de prioriteit van taken te beschrijven. Deze prioriteiten zijn verdeeld in vier groepen: \textit{Must have} is een vereiste van het project. Als een must-have niet is verwerkt in het eindresultaat kan het project worden gezien als gefaald. \textit{Should have} zijn punten die zeer interessant zijn voor het project, maar niet noodzakelijk voor de kern van het project en dus ook om het project te laten slagen. Punten binnen de \textit{Could have} zijn interessante toevoegingen voor bijvoorbeeld de gebruikerservaring, en kosten over het algemeen weinig tijd om toe te voegen. Vaak worden deze punten toegevoegd wanneer aan het einde nog tijd over is. \textit{Won't have} zijn ideeën voor in de toekomst. Deze punten zitten vanwege tijdgebrek vrijwel nooit in het eindproduct, en hebben vaak de minste impact in de werking van het eindresultaat van deze iteratie van het project.

\noindent\fbox{%
    \parbox{\textwidth}{%
        \textbf{Must have}
        \begin{itemize}
            \item Argumentatie die het concept ondersteunt
            \item Business Model Canvas
            \item Uitgewerkt en uitgeschreven businessplan
        \end{itemize}
        \textbf{Should have}
        \begin{itemize}
            \item Eerste stappen van realiseren businessplan uitgevoerd
            \item Werkend prototype voor intern gebruik
        \end{itemize}
        \textbf{Could have}
        \begin{itemize}
            \item Prototype in gebruik bij een testpubliek
            \item Eerste inkomsten
        \end{itemize}
        \textbf{Won't have}
        \begin{itemize}
            \item Volledig draaiend bedrijf
        \end{itemize}
    }%
}

% ---------------------------------------------------------------------------------------- %
\subsection{Onderwerp}
% Het onderwerp: wat is het verschijnsel, de technologie, het artefact, de persoon, enzovoort, waarover de paper gaat?
Binnen dit Supportive Narrative zal onderzoek gedaan gaan worden naar fysieke verkoop van content binnen de steeds meer digitale muziekindustrie. Door de digitalisering van de afgelopen 20 jaar is de verkoop van fysieke media sterk gedaald. \todo{Bron zoeken.} Dit komt door het gemak en de toegankelijkheid die digitale platformen hebben gemaakt. In plaats van een kast vol met CD's en LP's heeft een gebruiker toegang tot miljoenen nummers op een server. Het probleem hiervan is dat het individu hierdoor bijna geen waarde meer heeft.

Fysieke producten aan de andere kant hebben veel eigen waarde. Ze nemen fysieke ruimte in beslag, hebben productiekosten, een beperkte oplage, en beschadigen over de tijd. Nog belangrijker hebben ze een emotionele waarde; de gebruiker zal zich bijvoorbeeld herinneren dat de CD op een bepaald festival is gekocht, of er staat een handtekening van de artiest geschreven op de album hoes van een LP.

Het is moeilijk om deze eigenschappen een-op-een toe te passen op digitale producten. Digitaal is niet uniek. Waar in het geval van reproductie van fysieke media vrijwel altijd de kwaliteit per reproductie omlaag gaat is het in computers mogelijk een bit-perfecte kopie te maken van het origineel. Helemaal met de innovaties in internet kan hoge kwaliteit muziek zonder vertraging te ervaren gestreamd worden.

Het onderwerp van dit SN zal zijn een evenwicht te vinden tussen deze twee soorten artefacten. Hierbij zal worden gezocht naar een combinatie van het gemak van digitale muziek, en de waarde van fysieke artefacten. De keuzes en onderbouwingen zullen worden beschreven in het SN en worden gebruikt om het businessplan van de NichePlayer te ondersteunen.

% ---------------------------------------------------------------------------------------- %
\subsection{Context}
% De context van het onderwerp: wat is het grotere kader?
Dit Supportive Narrative zal gaan over technologie in het distributieproces van muziek. De focus daarbinnen richt zich op het evenwicht tussen het gemak van digitaal en de waarde van het fysieke artefact.

% ---------------------------------------------------------------------------------------- %
\subsection{Relevantie}
% Relevantie: waarom is het onderwerp van belang voor anderen?
De concepten en de dienst die in dit SN wordt beschreven zijn relevant binnen de distributieprocessen van de muziek industrie. Het biedt beginnende en kleine artiesten en producers de kans een grotere inkomsten stroom te genereren uit de verkoop van hun muziek. Dit zou momenteel niet mogelijk zijn met bestaande diensten als Spotify en Bandcamp.

% ---------------------------------------------------------------------------------------- %
\subsection{Motivatie}
% Motivatie: waarom is het onderwerp relevant voor de eigen praktijk?
\begin{itemize}
    \item Artiesten helpen
    \item Business opzetten
\end{itemize}
