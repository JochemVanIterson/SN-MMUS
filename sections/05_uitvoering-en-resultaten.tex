\section{Uitvoering en resultaten}
% In dit hoofdstuk wordt de daadwerkelijke uitvoering van de practice based research beschreven, waarbij de in het hoofdstuk methode beschreven aanpak wordt gevolgd. De verhouding tussen praktijk in de vorm van projecten en onderbouwing kan flink uiteenlopen, van 80/20% tot 20/80%. De thesis is daarbij niet alleen een droog projectverslag. De nadruk moet op de verantwoording liggen: reflectie op de relatie tussen de gevolgde aanpak aanpak en resultaten.

\subsection{Onderbouwen aannames}

\subsubsection*{Vernieuwd luistergedrag}

\begin{quotebox}
Door het vernieuwde luistergedrag binnen streamingdiensten heeft het album als distributievorm zijn individuele waarde verloren
\end{quotebox}
\begin{itemize}
  \item Er wordt niet meer naar albums geluisterd, maar meer naar individuele nummers in playlists
\end{itemize}

\subsubsection*{Waarde verschoven naar platform}
\begin{quotebox}
De waarde zit op dit moment niet meer in de muziek, maar het platform die toegang biedt tot de voor de luisteraar haast onbeperkte hoeveelheid aan content
\end{quotebox}
\begin{itemize}
  \item Luisteraars betalen niet voor de muziek, maar voor de toegang. Ze hebben vaak geen idee hoe het geld verdeeld wordt onder de artiesten.
\end{itemize}

\subsubsection*{Waarde zit in community}
\begin{quotebox}
De waarde van muziek zit in de community die wordt gevormd rondom de muziek
\end{quotebox}
\begin{itemize}
  \item Merchandise is erg belangrijk. Het is een manier om te laten zien dat je bij de community hoort
\end{itemize}

\subsubsection*{Beginnende artiesten niet rendabel}
\begin{quotebox}
Doordat beginnende artiesten op streamingdiensten direct concurreren met artiesten op professioneel niveau is het moeilijk om door te breken. Dit is slecht voor de diversiteit binnen de muziekindustrie
\end{quotebox}
\begin{itemize}
  \item 
\end{itemize}

\subsubsection*{Muziek raakt kwijt in geheel}
\begin{quotebox}
Hoewel het steeds makkelijker is om muziek te uploaden, zorgt dit er ook voor dat de muziek kwijt raakt in het grote geheel
\end{quotebox}
\begin{itemize}
  \item Iedereen kan muziek maken en dit makkelijk uploaden naar streamingdiensten. Hierdoor is er een overvloed aan muziek. Het is moeilijk om op te vallen tussen al deze muziek.
  \item Muzikanten maken muziek waarmee ze makkelijker op playlists komen. Hierdoor wordt er minder geëxperimenteerd met muziek. Resultaat is dat de muziek meer van hetzelfde is.
\end{itemize}

\subsection {Uitwerken businessplan}

\todo{Notes}
\begin{itemize}
  \item Daadwerkelijke businessplan uitgeschreven in bijlage, samen met presentatie/pitch
\end{itemize}
% 
% \subsection*{OUTLINE}
% In dit hoofdstuk wordt verslag gegeven van het proces van het onderzoek. Het businessplan zal hier worden uitgewerkt en onderbouwd. Daarnaast zal het ook worden gebruikt om het prototype te kunnen realiseren en testen bij een testpubliek. Dit testen zal worden geprobeerd te doen tijden tijdens het onderzoek, maar dat is afhankelijk van de beschikbare tijd.
