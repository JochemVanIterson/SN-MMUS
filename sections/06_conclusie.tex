\section{Conclusie en reflectie}
% In dit hoofdstuk vindt de terugkoppeling naar zowel de uitvoering van het werk (hfs 4) en het oorspronkelijke doel (hfs 1) plaats. Wanneer alle onderdelen van de paper - doel, vraagstelling, methode, uitvoering - goed op elkaar aansluiten - een goede 'logical construct' -, zal er altijd een zinvolle conclusie te maken zijn. Ook wanneer de research, tegen de verwachting in, niet veel oplevert is dat van belang als conclusie: wanneer de constructie van het geheel goed is geweest, zal iemand die zich met hetzelfde onderwerp gaat bezighouden de research kunnen volgen en een alternatieve aanpak kunnen ontwikkelen. In dit hoofdstuk komen dan ook aanbevelingen aan de orde: de losse draden die in toekomstige projecten opgepakt kunnen worden. Een paper of thesis bevat naast bovenstaande hoofdstukken altijd een (engelstalige) samenvatting (abstract) aan het begin, en een standaard opgemaakt literatuurlijst (Harvard style) aan het einde. Bij een meer uitgebreide paper of thesis kan allerlei ondersteunend materiaal in bijlages toegevoegd

% conclusie is reflectie, evaluatie en toekomst plannen
Zoals in mijn inleiding beschreven was mijn doel voor dit Supportive Narrative om een contextueel onderzoek te doen bij het ontwikkelen van de NichePlayer. Dit wilde ik doen door middel van het onderbouwen van een aantal aannames. Daarnaast wilde ik een prototype ontwikkelen om te laten zien dat het concept van de NichePlayer werkt. In dit hoofdstuk zal ik reflecteren op het proces en de resultaten van het onderzoek en het project.

\subsection{Businessmodel}
Het schrijven van een businessplan en deze scriptie heeft er voor gezorgd dat ik het concept van de NichePlayer goed heb uitgewerkt. Hierdoor kon ik doelgericht te werk gaan tijdens het ontwikkelen van de NichePlayer, en is de ontwikkeling bovendien vele male sneller gegaan.

Hoewel de aannames die ik had geformuleerd uiteindelijk niet allemaal onderbouwd zijn met bronnen, heb ik door het duidelijk formuleren en beschrijven wel stappen gezet in het verantwoorden van mijn keuzes.

Het businessplan is, net als de rest van het project, een levend document, en zal in de toekomst nog verder uitgewerkt worden.

\subsubsection*{Rechten}
Tijdens het opzetten van het businessplan heb ik niet nagedacht over het verkrijgen van rechten van de muziek. Het blijkt noodzakelijk te zijn om rechten te hebben voor de muziek die wordt afgespeeld binnen het systeem. Dit is complex omdat de NichePlayer een decentraal platform is, en de muziek dus niet vanaf één centrale plek wordt afgespeeld. Artiesten moeten de rechten daarom zelf verwerken in de dienst. Daarbij komt ook dat rechten vaak maar in een bepaald gebied of land geldig zijn, waardoor voor iedere luisterregio de rechten opnieuw moeten worden geregeld. Dit is een complex proces wat meegenomen moet worden in het businessplan.

Ik ben tot dit inzicht gekomen door de Netflix serie 'The Playlist'. Deze Zweedse serie gaat over het oprichten van Spotify. Dit verhaal wordt uitgelicht vanuit de verschillende perspectieven van de oprichters en werknemers. Hoewel de serie gaat over een waargebeurd verhaal, is duidelijk dat het is aangepast om te werken als serie. Toch zijn er een hoop interessante punten die worden aangehaald, waaronder het verkrijgen van rechten.

\subsubsection*{Pay per stream}
Hoewel ik had nagedacht over het bieden van zo veel mogelijk betaalopties aan de luisteraar, had ik niet nagedacht hoe deze betalingen vervolgens zouden verlopen. Na gesprekken met klasgenoten kwam ik er achter dat de betaaloptie pay-per-stream niet zo makkelijk is als gedacht. Waar de meeste andere opties relatief grote bedragen hebben die vooraf betaald kunnen worden gaat het bij pay-per-stream om een hele hoop kleine bedragen. Naast dat dit voor de artiest veel administratie en transactie kosten oplevert, is het ook voor de luisteraar niet prettig om elke keer dat een nummer wordt afgespeeld een betaling te moeten doen. Het is beter om achteraf, aan het einde van de maand, een factuur of betaling te sturen met daarop alle nummers die zijn afgespeeld. Er kan ook worden gekeken naar een saldo systeem waarbij de luisteraar eerst geld op het account zet en vervolgens van dat saldo wordt afgeschreven.

\subsection{Prototype}
Hoewel ik de wens had veel verder te zijn met het ontwikkelen van het prototype, is nu duidelijk dat het vooral een proof-of-concept is. Er zijn nog een hoop features die ik graag zou willen toevoegen, maar waar te weinig tijd voor was om te implementeren. Met de huidige staat van het prototype is het echter wel mogelijk om een demo te geven van het concept. Het laat hiermee zien dat de muziekindustrie ook anders kan, en dat het mogelijk is om zelf een streamingdienst te hebben.

Hoewel het prototype in erg korte tijd is opgezet ben ik blij met de snelheid waarop ik het heb kunnen maken en de kwaliteit van het resultaat. Dit laat zien dat het framework wat ik nu heb ontwikkeld erg flexibel is en makkelijk te verwerken in andere projecten. Door gebruik te maken van plugins kan ik aan meerder projecten tegelijk werken, en kan ik de plugins ook in andere projecten gebruiken.

Een demo van het prototype is te vinden op \url{https://player-demo.nicheplayer.nl/}. Een demo account is aangemaakt met gebruikersnaam en wachtwoord 'demo'. Omdat het project nog steeds erg in ontwikkeling is heb ik screenshots en een video gemaakt van de huidige status. De screenshots zijn te vinden in bijlage \ref{bijlage:screenshots_demo}. De content die op de demo staat is geen eigen muziek maar muziek uit mijn netwerk. Dit laat goed zien dat het een dynamisch platform is die niet forceert tot één genre of stijl.

\subsubsection*{Scaling en toekomstbestendigheid}
Er is nagedacht over scaling en toekomstbestendigheid van het systeem. Door het decentrale karakter van het systeem (iedere artiest heeft zijn eigen player) is het systeem niet direct belastend voor één server. Dit wordt echter wel een probleem wanneer de artiest bekender wordt en er veel gebruikers tegelijk actief zijn. Er wordt dan veel data tegelijk van de server opgehaald wat een probleem kan zijn bij slechte bandbreedte en CPU.

Het probleem van scaling was de reden dat de vorige iteratie van de NichePlayer niet goed werkte. Door de gebruikte libraries beter te configureren en te optimaliseren is het limiet van de server flink verbeterd. Dit is echter nog geen volledige oplossing voor het probleem. Er zijn verschillende (software) oplossingen die dit probleem kunnen oplossen zoals load balancing en het transcoderen van de media. Deze oplossingen zijn echter lastig toe te passen zijn wanneer de artiest zelf de NichePlayer host omdat dit erg platform afhankelijk is. 

Er is bewust gekozen om de NichePlayer te ontwikkelen voor kleine artiesten, waardoor het probleem van scaling voorlopig nog niet aan de orde is. Toch is het belangrijk om hier rekening mee te houden bij het verder ontwikkelen van het systeem.

\subsubsection*{Gebruikerservaring en automatisering}
Na het presenteren van de demo aan verschillende klasgenoten en andere geïnteresseerden blijkt het luisteraar deel van de NichePlayer intuïtief te werken. Er was weinig uitleg nodig om een account aan te maken en muziek te luisteren. Het concept van het platform was snel duidelijk, en de reacties waren positief. Verder was de User Interface duidelijk en overzichtelijk. Doordat er een taal-engine is ingebouwd kunnen ook niet-Nederlandse gebruikers de NichePlayer gebruiken. Het is makkelijk om meerdere talen toe te voegen, hoewel daarvoor wel de bronbestanden moeten worden aangepast. Het is mogelijk deze variabelen los te halen en in een apart bestand te zetten, zodat deze makkelijk kunnen worden aangepast.

Mensen waren verbaast over de snelheid, zeker aangezien de demo server momenteel op mijn studentenkamer staat en de kwaliteit van de bestanden die wordt afgespeeld erg hoog is. Een groot deel van de muziek is geüpload als FLAC, een lossless audioformaat. De bestanden zijn hiermee erg groot (ongeveer 25MB voor een nummer van 3 minuten), maar worden vrijwel direct afgespeeld. Dit is mogelijk doordat de bestanden worden gestreamd, en niet eerst volledig worden gedownload. De bestanden worden in kleine stukjes opgehaald van de server en afgespeeld. Hierdoor is het mogelijk om een nummer te luisteren terwijl het nog wordt gedownload. In de toekomst is het idee om deze FLAC bestanden te transcoderen naar verschillende kleinere formaten waaronder MP3 en Ogg, zodat de luisteraar zelf kan kiezen welke kwaliteit hij wil luisteren en het systeem niet onnodig veel data hoeft te versturen. Dit is ook goed voor de scaling van het systeem.

Bij het ontwikkelen van het prototype is er vooral gekeken naar de gebruikerservaring van de luisteraar. De administratie kant van het systeem is erg uitgebreid, maar daardoor ook erg complex. Het aanmaken van een album is momenteel erg veel werk, alle instellingen moeten handmatig worden ingevuld. Dit is een van de punten die ik in de toekomst wil verbeteren. Omdat de ID3 tags zijn opgeslagen in de database kan ik deze gebruiken om automatisch velden in te vullen. Hiermee kan ik het aanmaken van een album sterk versimpelen.

Er lijkt nog een probleem te zijn met het verversen van de gebruiker sessies. Wanneer een gebruiker een tijd niet actief is raakt de sessie verlopen en moet deze gerefreshed worden. Dit lijkt echter niet te gebeuren wanneer de gebruiker een nummer aan het luisteren is, en is zeker een probleem bij audiobestanden langer dan 10 minuten. Aangezien het afspelen gebruik maakt van streaming is het mogelijk dat een bestand maar half wordt opgehaald, waarna de sessie verlopen is en het afspelen crashed. Dit probleem is simpel op te lossen door om de zoveel tijd de sessie te verversen. Het is niet mogelijk de duur van de sessie te verlengen, omdat dit een beveiligingsrisico is.

\subsection{Reflectie Proces}

\subsubsection*{Project}
Ik heb erg lang uitgesteld om te beginnen met het daadwerkelijke product van de NichePlayer. Dit kwam voornamelijk door een van de projecten die er voor aan de basis lag: de U.F.O.-App. Hoewel ik het zelf prettig vond om doelbewust niet te werken aan het project maar het einddoel constant wel in de gaten te houden, was het voor mensen die mij begeleidde niet altijd duidelijk wat ik aan het doen was. Voor mij zelf was duidelijk dat er erg veel overeenkomsten waren, maar ik bleek dit niet goed over te kunnen brengen aan anderen die er meer buiten stonden.

Het werken aan de U.F.O.-App heeft een aantal keer vertraging opgelopen. Door externe factoren kon ik soms lange tijd niet aan het project werken. Daarnaast zat ik op een gegeven moment vast in een bepaald programmeer probleem wat ik niet opgelost kreeg. Door een tijd helemaal te stoppen (harde reset) en daarna weer opnieuw te beginnen heb ik het probleem uiteindelijk op kunnen lossen. Dit heeft echter wel een hoop extra tijd gekost, waardoor het implementeren van de NichePlayer vertraging opliep.

In de toekomst wil ik nog steeds deze methode toepassen. Het sluit goed aan bij mijn werkproces en ik kom nog steeds tot goede resultaten. Wel wil ik in de toekomst beter communiceren over wat ik aan het doen ben, waar ik heen wil, en waarom ik bepaalde keuzes maak. Deze keuzes zijn nu beschreven in dit SN, wat eigenlijk te laat is voor goede begeleiding en feedback. Een manier waarop ik dit kan doen is meer structureel geplande proces updates, het actief bijhouden van een logboek en het beter bijhouden van mijn todo-lijst. Wat al leek te helpen was het vooraf definiëren van een einddoel.

\subsubsection*{Supportive Narrative}
Ik heb dit Supportive Narrative geschreven in de programmeertaal LaTeX. Deze taal schrijft in platte tekst, en exporteert vervolgens naar andere formats zoals PDF en DOCX. Dit heeft voor mij meerdere voordelen gehad. Door te schrijven in platte tekst hoef ik tijdens het schrijven niet na te denken over de opmaak aangezien deze automatisch voor mij wordt gegenereerd. Daarnaast is platte tekst makkelijk op te nemen in een Git repository, die ik heb gebruikt voor versiebeheer van zowel mijn SN als het project.

Hoewel ik al in mei 2022 de structuur van het SN heb opgezet en in oktober 2022 een eerste stuk van de introductie schreef, ben ik pas begin april 2023 écht begonnen met schrijven. Hoewel ik in de tussentijd een outline heb bijgehouden en veel heb nagedacht over het onderwerp en het project, heb ik het daadwerkelijk schrijven van het SN te lang uitgesteld. Hierdoor heb ik mijn plannen moeten bijstellen en heb ik onnodig veel stress veroorzaakt bij mijzelf. 

Ik heb gemerkt dat ik mij slechts kan focussen op één grote taak tegelijk, en snel motivatie verlies als ik in die taak niet genoeg voortgang boek. Dit heb ik tijdens mijn bachelor al opgemerkt, en tijdens mijn master proberen aan te pakken. Binnen mijn softwareontwikkeling-werk heb ik dit opgelost door meerdere projecten tegelijk te hebben, en hier tussen te wisselen wanneer ik vastloop. Tijdens dit SN heb ik het schrijven gezien als een van deze zij-projecten, wat mij heeft geholpen met het volhouden van het werken aan het SN. Omdat schrijven echter niet mijn sterkste kant is heb ik vaak andere projecten toch voorrang gegeven boven het SN. Hierdoor kwam ik uiteindelijk toch in tijdsnood en heb ik het SN in een korte tijd moeten schrijven.

Wat mij helpt is het hebben van een deadline, in combinatie met een backup plan. Ik heb meerdere keren planningen gemaakt, zowel voor het schrijven als het ontwikkelen van het prototype. Toch week ik hier constant vrij snel van af, ook omdat tijd inschatten binnen software ontwikkeling niet altijd mogelijk is door onverwachte bugs. Deadlines geven mij een motivatie om te blijven werken. Daarbij verminderd het hebben van een backup bij mij het stress gevoel omdat ik weet dat ik nog een extra kans heb om het af te maken. Uiteindelijk blijk ik de backup vaak niet nodig te hebben, maar het zorgt er wel voor dat ik mij minder druk maak over het halen van de deadline.

Een onderdeel van het SN wat ik meer had willen doen is onderbouwen met bronnen. Omdat deze manier van onderzoek voor mij nieuw en onbekend is heb ik het lange tijd uitgesteld. Dit heeft onder andere te maken met dat ik nog niet goed weet hoe ik goede bronnen kan vinden, waardoor ik snel ontmoedigd raak wanneer ik niet direct vind wat ik nodig heb. Daarnaast lees ik niet snel, waardoor het lezen van een artikel mij veel energie kost. Ik ben door dit SN echter wel beter geworden in het vinden van bronnen, wat zeker in de toekomst goed van pas zal komen. Zo weet ik nu beter welke zoektermen ik moet gebruiken, ben ik bekend met verschillende zoekmachines zoals Google Scholar en het gebruiken van de bronnenlijst binnen Wikipedia, en kan ik beter filteren op relevantie van artikelen door het lezen van abstracts.

Ik ben blij met het resultaat, zeker gezien de toch geringe hoeveelheid tijd die ik heb gestoken in het schrijven er van. Ik heb gedurende de afgelopen twee jaar veel nagedacht en ideeën uitgewerkt in mijn hoofd. Uiteindelijk heb ik mijn soms toch chaotische en onsamenhangende gedachtegang weten te vertalen naar meer concrete tekst, en daarbij voor mijzelf de kern weten te vinden van mijn concept en project.

\subsection{Toekomst}
Zoals eerder genoemd is er nog veel werk te doen aan de NichePlayer, zowel op het gebied van het prototype, als het uitwerken van het businessplan.

Daarnaast merk ik dat er steeds meer extra werk komt kijken bij het opzetten van het bedrijf. Ik begin tegen de grenzen van mijn kennis aan te lopen en zal partners moeten vinden die mij kunnen helpen met de onderdelen waar ik zelf niet goed in ben. Dit zijn onder andere onderwerpen als muziekrechten, sales en marketing. Ik blijf een software ontwikkelaar, en zal mij daarom ook voornamelijk blijven richten op het ontwikkelen van de software.

De status van het project is nog niet genoeg om een volwaardig bedrijf mee op te starten en direct inkomen te hebben. Mijn plan is om de komende tijd een parttime baan te vinden binnen de software ontwikkeling. Daarnaast wil ik aan de NichePlayer en andere eigen projecten blijven werken, en het uiteindelijk uitbouwen tot een volwaardig bedrijf.

De reden dat ik hiervoor kies is omdat ik tijdens mijn stage in jaar 3 van mijn bachelor heb gemerkt dat een volledig fulltime baan mij niet ligt. Ik merkte dat ik door constant met dezelfde problemen bezig te zijn creatief uitgedoofd raakte en steeds minder in staat was om nieuwe oplossingen te bedenken en gemotiveerd naar werk te gaan. Door niet fulltime maar parttime te werken heb ik zekerheid van inkomen, en kan ik daarnaast aan mijn eigen projecten werken. Dit geeft mij de vrijheid om te werken aan wat ik wil en kan ik hierbij nieuwe technologieën leren kennen en mij blijven ontwikkelen.

\subsection{Conclusie}
Voor mij is het onderzoek geslaagd. Ik heb een werkend prototype en een veel beter beeld over de context waarin in de NichePlayer zal gaan zitten. Er is nog veel werk te doen, maar ik heb een goede basis na mijn opleiding verder te werken.
