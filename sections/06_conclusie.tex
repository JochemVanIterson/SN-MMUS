\section{Conclusie en reflectie}
% In dit hoofdstuk vindt de terugkoppeling naar zowel de uitvoering van het werk (hfs 4) en het oorspronkelijke doel (hfs 1) plaats. Wanneer alle onderdelen van de paper - doel, vraagstelling, methode, uitvoering - goed op elkaar aansluiten - een goede 'logical construct' -, zal er altijd een zinvolle conclusie te maken zijn. Ook wanneer de research, tegen de verwachting in, niet veel oplevert is dat van belang als conclusie: wanneer de constructie van het geheel goed is geweest, zal iemand die zich met hetzelfde onderwerp gaat bezighouden de research kunnen volgen en een alternatieve aanpak kunnen ontwikkelen. In dit hoofdstuk komen dan ook aanbevelingen aan de orde: de losse draden die in toekomstige projecten opgepakt kunnen worden. Een paper of thesis bevat naast bovenstaande hoofdstukken altijd een (engelstalige) samenvatting (abstract) aan het begin, en een standaard opgemaakt literatuurlijst (Harvard style) aan het einde. Bij een meer uitgebreide paper of thesis kan allerlei ondersteunend materiaal in bijlages toegevoegd

% conclusie is reflectie, evaluatie en toekomst plannen

\subsection{Businessmodel}
\begin{todolist}
  \item Het schrijven van een businessplan (en deze scriptie) zorgt er voor dat ik meer doelgericht te werk ga tijdens het ontwikkelen. Andersom geeft het ontwikkelen nieuwe input voor het businessplan.
  \item Hoewel de aannames niet allemaal zijn onderbouwd met bronnen, heb ik wel stappen gezet in het verantwoorden van mijn keuzes.
  \item Het businessplan is, net als de rest van het project, een levend document, en zal in de toekomst nog verder uitgewerkt worden. 
\end{todolist}

\subsection{Prototype}
\begin{todolist}
  \item Het prototype is nu vooral een proof-of-concept. Het laat zien dat de muziekindustrie ook anders kan, en dat het mogelijk is om zelf een streamingdienst te hebben.
  \item Demo screenshots zijn te vinden in bijlage \ref{bijlage:screenshots_demo}.
  \item Demo video moet nog gemaakt worden.
  \item Ik ben blij met de snelheid waarop ik het prototype heb kunnen maken. Dit laat zien dat het framework wat ik nu heb ontwikkeld erg flexibel is en makkelijk te verwerken in andere projecten.
  \item Het prototype is uiteraard nog lang niet af. Er zijn een hoop nuttige features die ik graag zou willen toevoegen, maar waar te weinig tijd voor was om te implementeren.
\end{todolist}

\subsection{Reflectie Proces}

\subsubsection*{Project}
Ik heb erg lang uitgesteld om te beginnen met het daadwerkelijke product van de NichePlayer. Dit kwam voornamelijk door een van de projecten die er voor aan de basis lag: de U.F.O.-App. Hoewel ik het zelf prettig vond om doelbewust niet te werken aan het project, maar het einddoel constant wel in de gaten te houden, was het voor de begeleiders niet altijd duidelijk wat ik aan het doen was. Voor mij zelf was duidelijk dat er erg veel overeenkomsten waren, maar ik bleek dit niet goed over te kunnen brengen aan anderen die er meer buiten stonden.

Het werken aan de U.F.O.-App heeft een aantal keer vertraging opgelopen. Door externe factoren kon ik soms lange tijd niet aan het project werken. Daarnaast zat ik op een gegeven moment vast in een bepaald programmeer probleem wat ik niet opgelost kreeg. Door een tijd helemaal te stoppen (harde reset) en daarna weer opnieuw te beginnen, heb ik het probleem uiteindelijk op kunnen lossen. Dit heeft echter wel een hoop extra tijd gekost, waardoor het implementeren van de NichePlayer vertraging opliep.

In de toekomst wil ik nog steeds deze methode toepassen. Het sluit goed aan bij mijn werkproces, en ik kom nog steeds tot goede resultaten. Wel wil ik in de toekomst beter communiceren over wat ik aan het doen ben, waar ik heen wil, en waarom ik bepaalde keuzes maak. Deze keuzes zijn nu beschreven in dit SN, wat eigenlijk te laat is voor goede begeleiding en feedback.

\subsubsection*{Supportive Narrative}
\begin{todolist}
  \item SN schrijven in Latex en GIT is net als tijdens mijn bachelor SN een goede keuze geweest. De tekstbestanden zijn goed geordend, versiebeheer is makkelijk en ik hoef mij niet bezig te houden met opmaak.
\end{todolist}

\todo{Schrijven conclusie en reflectie}
