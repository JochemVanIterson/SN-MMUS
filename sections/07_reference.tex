\bibliography{bibliography}

\subsection*{The Internet Archive}
Veel bronnen die ik heb gebruikt zijn te vinden op internet. Omdat ik niet zeker weet of deze bronnen altijd beschikbaar blijven heb ik ze opgeslagen in The Internet Archive. Dit is een website die een kopie maakt van een website op een bepaald moment. Deze kopie blijft dan beschikbaar op The Internet Archive. Wanneer de originele website niet meer beschikbaar is, kan de kopie gevonden worden door naar \url{https://archive.org/web/} te gaan en de URL van de originele website in te voeren.

\subsection*{AI en Prompt vermelding}
Bij het schrijven van dit Supportive Narrative is gebruik gemaakt van AI. Ik heb de tool Copilot van GitHub gebruikt als predictive-text hulp. Deze AI wordt normaal gebruikt als hulp tijdens het programmeren, maar omdat ik mijn Supportive Narrative in LaTeX schrijf kan ik deze tool ook gebruiken bij het schrijven van teksten.

Ik heb soms moeite bij het vinden van de juiste verwoording. Copilot raadt bij een stuk tekst wat ik al heb geschreven wat het vervolg er op moet gaan zijn. Negen van de tien keer komt hier pure onzin uit (veel herhalingen of incorrecte tekst), maar de paar keer dat het wel lukt is het erg fijn! De uitkomst van de AI pas ik grotendeels aan, waardoor het mijn eigen tekst wordt.

Het is bij deze AI niet mogelijk om aan prompt vermelding te doen omdat het verder schrijft bij bestaande tekst. In theorie is de tekst zelf dus de prompt. 
