\section{Methode}
\lipsum[5]

% De hoofdstukken introductie en critical review leiden in combinatie tot een vraagstelling. Deze vraagstelling moet goed in lijn liggen met het doel van het research. De vraagstelling en daarmee samenhangende definities bakenen het research verder af. In dit hoofdstuk wordt vervolgens de methode beschreven waarmee deze vraag wordt beantwoord. Van belang is dat de methode goed moet aansluiten op het doel en de vraagstelling, zodat er een logische samenhang is. Uitgangspunt is vaak een model - een al bestaand model of eigen voorstel - dat gebruikt wordt als 'bril' om naar het onderwerp te kijken. Er wordt een keus gemaakt uit een aantal methodes: participant observation, case studies, interviews, implementeren van in literatuur gevonden theorie, experimenten in de eigen praktijk, enzovoort. Omdat het hier om practice based research gaat is de eigen praktijk altijd leidend bij het kiezen van een methode. Het is belangrijk dat de methode goed aansluit bij het doel en de vraagstelling van het stuk.
