\section{Critical Review}
\captionsetup[figure]{font=small,labelfont=bf}
\lipsum[4]

% Dit hoofdstuk is een opsomming van dat wat er al bekend is over het in de introductie beschreven onderwerp. Zeker bij practice based research gaat dit niet alleen om academische literatuur. Ook repertoire in de vorm van al bestaand werk, maakprocessen, contexten, technologieën, enzovoort is relevant, net als literatuur die niet strict academisch is zoals artikelen uit vakbladen, documentaires, enzovoort. In het critical review hoofdstuk wordt het eerder geïntroduceerde onderwerp in verband gebracht met relevante literatuur en repertoire.

\begin{itemize}
    \item \url{https://web.archive.org/web/20011205101807/http://uk.eurorights.org/issues/cd/bad/}
    \item \url{https://www.digitaltrends.com/music/the-history-of-the-cds-rise-and-fall/}
    \item Philips history of the CD \begin{itemize}
        \item \url{https://www.philips.com/a-w/research/technologies/cd/beginning.htm}
        \item \url{https://www.philips.com/a-w/research/technologies/cd/technology.htm}
        \item \url{https://www.philips.com/a-w/research/technologies/cd/black-giants.htm}
        \item \url{https://www.philips.com/a-w/research/technologies/cd/introduction.htm}
        \item \url{https://www.philips.com/a-w/research/technologies/cd/jewel-case.htm}
        \item \url{https://www.philips.com/a-w/research/technologies/cd/cd-family.htm}
    \end{itemize}
    \item \url{https://dutchaudioclassics.nl/the-emergence-of-the-compact-disc/}
    
\end{itemize}