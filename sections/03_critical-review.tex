\section{Critical Review}
\captionsetup[figure]{font=small,labelfont=bf}
% Dit hoofdstuk is een opsomming van dat wat er al bekend is over het in de introductie beschreven onderwerp. Zeker bij practice based research gaat dit niet alleen om academische literatuur. Ook repertoire in de vorm van al bestaand werk, maakprocessen, contexten, technologieën, enzovoort is relevant, net als literatuur die niet strict academisch is zoals artikelen uit vakbladen, documentaires, enzovoort. In het critical review hoofdstuk wordt het eerder geïntroduceerde onderwerp in verband gebracht met relevante literatuur en repertoire.

\subsection*{TODO}

\begin{itemize}
    \item Markt en 'echte' wereld
    \begin{itemize}
        \item Critical review naar bestaande media (Phonograph, LP, tape, CD, MP3, Torrent/P2P, Streaming)
        \item Korte technische uitleg wat het is
        \item Disruptieve werking van een medium op zijn voorganger
        \item Welk probleem heeft het proberen op te lossen
    \end{itemize}
    \item Eigen werk, directe context
    \begin{itemize}
        \item Korte geschiedenis van de NichePlayer (vorige iteraties)
        \item TapTapes
    \end{itemize}
\end{itemize}

% Inleiding
In dit critical review zal ik een onderzoek doen naar bestaande media en diensten voor muziek distributie. Hierbij zal ik beschrijven wat de disruptieve werking was van het beschreven medium ten opzichte van zijn voorganger, en wat het probleem was wat het medium probeerde op te lossen. Deze analyse is relevant omdat het project van dit Supportive Narrative (de NichePlayer) als ultiem doel heeft een nieuw medium te worden.

% ---------------------------------------------------------------------------------------- %
\subsection{Analyse bestaande media}
In dit onderzoek kan in theorie terug worden gegaan tot in de oudheid. Het doorgeven en distribueren van muziek is een onderdeel van cultuur, en daarmee een iets wat ons mens maakt. Om een kaderen te stellen wordt gefocust op technologie uit de afgelopen 150 jaar, beginnende bij de fonograaf.

\subsubsection*{Fonograaf}
De fonograaf (phonograph in Engels) is een van de eerste distributie media die het geluid zelf draagt ten opzichte van een beschrijving van de muziek. 

\begin{itemize}
    \item Opvolger op live muziek en zelf spelen
    \item \todo{Opzoeken speeltijd}
    \item \todo{Opzoeken exacte datum uitvinding}
    \item Wasrol
    \item Muziek hoe 'het zou moeten klinken'
    \item 
\end{itemize}

\subsubsection*{Grammofoon en iteraties}
\begin{itemize}
    \item Platte plaat van vinyl
    \item Makkelijker in opslag
    \item Makkelijker in (massa) productie
    \item Veel iteraties: langere platen, stereo, kwaliteit verbetering, etc.
\end{itemize}

\subsubsection*{Tape}
\begin{itemize}
    \item 
\end{itemize}

\subsubsection*{CD}
\begin{itemize}
    \item 
\end{itemize}

\subsubsection*{MP3 en digitale opkomst}
\begin{itemize}
    \item 
\end{itemize}

\subsubsection*{Internet (Torrents en Peer2Peer)}
\begin{itemize}
    \item 
\end{itemize}

\subsubsection*{Streaming}
\begin{itemize}
    \item 
\end{itemize}

% ---------------------------------------------------------------------------------------- %
\subsection{Concurrerende projecten}
\subsubsection*{TapTapes}
De NichePlayer is niet alleen in zijn soort. Er zijn veel vergelijkbare producten die het in de inleiding benoemde probleem proberen op te lossen. Een van deze diensten is TapTapes, een start-up uit Saarbrücken, Duitsland. Toevallig is dit het bedrijf waar ik in mijn bachelor stage heb gelopen. Het concept van de NichePlayer bestond toen al enkele jaren, maar heeft door de stage wel een boost gekregen.

