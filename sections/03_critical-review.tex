\section{Critical Review}
\captionsetup[figure]{font=small,labelfont=bf}
% Dit hoofdstuk is een opsomming van dat wat er al bekend is over het in de introductie beschreven onderwerp. Zeker bij practice based research gaat dit niet alleen om academische literatuur. Ook repertoire in de vorm van al bestaand werk, maakprocessen, contexten, technologieën, enzovoort is relevant, net als literatuur die niet strict academisch is zoals artikelen uit vakbladen, documentaires, enzovoort. In het critical review hoofdstuk wordt het eerder geïntroduceerde onderwerp in verband gebracht met relevante literatuur en repertoire.

% NOTES
% - Critical review naar bestaande media (Phonograph, LP, tape, CD, MP3, Torrent/P2P, Streaming)
% - Korte technische uitleg wat het is
% - Disruptieve werking van een medium op zijn voorganger
% - Welk probleem heeft het proberen op te lossen

% Inleiding
In dit critical review zal ik een deskresearch onderzoek doen naar bestaande media en diensten voor muziek distributie. Hierbij zal worden beschreven wat de disruptieve werking was van het medium ten opzichte van zijn voorganger, en wat het probleem is wat het medium probeerde op te lossen. Tot slot zal worden beschreven waarom het medium relevant is voor dit Supportive Narrative en het onderliggende project, de NichePlayer.

% Uitleggen disruptief effect
Nieuwe media worden vaak uitgevonden in een actie-reactie proces. In het oude medium blijkt na gebruik een tekortkoming te zitten, wat door middel van het nieuwe medium wordt opgelost. Dit proces is niet altijd intentioneel. Sommige innovaties in de geschiedenis hebben geleid tot een grotere hoeveelheid piracy, veranderingen in de muziek zelf, en hoe men er naar luistert.

\todo{Fact-checken van alle onderstaande uitspraken}

% ---------------------------------------------------------------------------------------- %
\subsection{Analyse bestaande media}
In dit onderzoek kan in theorie terug worden gegaan tot in de oudheid. Het doorgeven en distribueren van muziek is een onderdeel van cultuur, en daarmee iets wat ons mens maakt. Om een kaderen te stellen wordt gefocust op distributiemedia die klinkend geluid (i.e. geluidsgolven) dragen, beginnende bij de fonograaf.

\subsubsection*{Fonograaf}
% Beschrijving medium
De fonograaf (phonograph in Engels) is een van de eerste distributie media die het geluid zelf draagt ten opzichte van een beschrijving van de muziek.

De geluidsgolven worden opgeslagen door met een naald in een wassen rol te krassen. De diepte van de kras staat gelijk aan de uitslag van het membraan. Tijdens het afspelen wordt het proces omgedraaid; een naald glijdt door de krassen en brengt hiermee het membraan in beweging. Het geluid wordt vervolgens versterkt door een hoorn.

% Klinkende muziek in huis zonder instrument
Door de uitvinding van de fonograaf werd het mogelijk om klinkende muziek in huis te hebben, zonder dat hier een instrument voor bespeeld hoefde te worden.

% Alle soorten muziek beschikbaar
Hiermee is het dus ook mogelijk om een opname van een heel orkest af te spelen binnen huis, iets wat voor de uitvinding niet kon vanwege de ruimte.

% Minder focus op eigen interpretatie
Mensen konden hiermee ook 'perfecte' versies van de muziek beluisteren. In plaats van eigen interpretaties van bladmuziek wordt geluisterd naar een enkele uitvoering van een artiest.Niet alle aspecten van het bespelen van een instrument kunnen immers beschreven worden op papier, waardoor er altijd een eigen interpretatie wordt gespeeld door de muzikant.

% Beperking: maximale speeltijd
Een beperking ten opzichte van zijn de voorgangers is dat de fonograaf (en opvolgende iteraties) een maximale speeltijd hadden, waar geschreven muziek in principe een oneindige lengte kan hebben.

\todo{Relevantie NichePlayer}

% NOTES
% - Opvolger op live muziek en zelf spelen
% - Muziek 'hoe het zou moeten klinken'
% - Redelijk goedkope manier om (klinkende) muziek in je huis te krijgen
% - Beperking in speeltijd

\subsubsection*{Grammofoon en iteraties}
De grammofoonplaat is een iteratie op de wasrol. In plaats van een ronde rol is het geluid bij de grammofoon opgeslagen op een platte plaat. Het schrijven en afspelen werkte bij de eerste iteraties van de grammofoon nog steeds voornamelijk mechanisch.

In de latere iteraties van de grammofoonspeler werd meer electronica verwerkt waardoor de kwaliteit en volume van het geluid beter werd. Daarnaast werd het ook mogelijk om de muziek in stereo af te spelen.

Een groot voordeel van de grammofoon is het medium waarop de muziek is opgeslagen: een plaat ten opzichte van de wasrol. Platen zijn het makkelijker te massa-produceren omdat ze gedrukt kunnen worden vanuit een mal. Een wasrol moet individueel worden beschreven en kan hierdoor niet makkelijk in grote hoeveelheid geproduceerd worden. Platen zijn daarnaast ook makkelijker in opslag omdat ze plat zijn. Door het gebruik van vinyl bij latere platen blijven ze ook langer goed.

Net als de wasrol heeft een plaat een maximale speelduur vanwege zijn formaat. Door technische restricties zijn verschillende standaarden ontstaan voor de lengte van albums. De oudste grammofoonplaten hadden door hun formaat (30cm, 12"), naalddikte en toerental van 78rpm een maximale speelduur van ongeveer 5 minuten. Het toerental komt voort uit een optimale geluidskwaliteit bij de afstand tussen de groeven van een 12" plaat, en het gebruik van standaard naainaalden. Door technische verbeteringen werd het vanaf 1950 mogelijk om platen van 40 tot 50 minuten te drukken.

\todo{Relevantie NichePlayer}
Deze standaarden worden tegenwoordig nog steeds aangehouden, hoewel dit vanwege de nieuwe technologieën niet meer nodig is.

% NOTES
% - Na wasrol platte plaat van vinyl
% - Vinyl makkelijker in opslag
% - Vinyl makkelijker in (massa) productie
% - Veel iteraties: langere platen, stereo, kwaliteit verbetering, etc.

\subsubsection*{Tape}
% Beschrijving medium
Bij de Tape is het geluid opgeslagen op een magnetisch geladen strip.

\todo{Relevantie NichePlayer}

\begin{itemize}
    \item Kopiëren (met kwaliteitsverlies)
    \item Nieuwe Cultuur, Mixtapes
    \item Opnemen vanuit andere media zoals radio (piracy!)
\end{itemize}

\subsubsection*{CD}
De CD is het eerste digitale medium. Net als de grammofoonplaat wordt het geluid opgeslagen op een schijf door middel van putjes. Bij een CD wordt echter geen gebruik gemaakt van een naald, maar wordt de diepte van de groef gelezen door een laser.

Het grote voordeel van de CD is de kwaliteit.

Daarnaast gaat de kwaliteit van CD's over lange tijd niet verloren.

\todo{Relevantie NichePlayer}

\begin{itemize}
    \item Hoge kwaliteit
    \item Geen kwaliteit verlies na lange tijd opslag
    \item Hoewel het populair was heeft niemand meer een speler (voordeel digitaal)
    \item Bootlegging
\end{itemize}

\subsubsection*{Internet (Torrents en Peer2Peer)}

\todo{Relevantie NichePlayer}

\begin{itemize}
    \item Alleen mogelijk door opkomst MP3
    \item Onbeperkt kopiëren zonder kwaliteitsverlies
    \item Illegale toegang, wel lang wachten voor kwaliteit
    \item Downloaden lossen nummers
\end{itemize}

\subsubsection*{Streaming}

\todo{Relevantie NichePlayer}

\begin{itemize}
    \item Legale, onbeperkte toegang
    \item Ander luistergedrag: \begin{itemize}
        \item Album minder belangrijk dan playlists
        \item Drop uitgesteld tot na 1:00 ivm betaling
    \end{itemize}
    \item Meer muziek als singles
\end{itemize}

% ---------------------------------------------------------------------------------------- %
\subsection{Concurrerende projecten}
\subsubsection*{TapTapes}
De NichePlayer is niet alleen in zijn soort. Er zijn veel vergelijkbare producten die het in de inleiding benoemde probleem proberen op te lossen. Een van deze diensten is TapTapes, een start-up uit Saarbrücken, Duitsland. Toevallig is dit het bedrijf waar ik in mijn bachelor stage heb gelopen. Het concept van de NichePlayer bestond toen al enkele jaren, maar heeft door de stage meer voor gekregen.

\subsubsection*{NichePlayer}
\begin{itemize}
    \item Korte geschiedenis van de NichePlayer (vorige iteraties)
    \item Verschil met vorige iteraties NichePlayer, waarom was een nieuwe iteratie nodig?
\end{itemize}
