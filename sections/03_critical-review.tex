\section{Critical Review}
\captionsetup[figure]{font=small,labelfont=bf}
% Dit hoofdstuk is een opsomming van dat wat er al bekend is over het in de introductie beschreven onderwerp. Zeker bij practice based research gaat dit niet alleen om academische literatuur. Ook repertoire in de vorm van al bestaand werk, maakprocessen, contexten, technologieën, enzovoort is relevant, net als literatuur die niet strict academisch is zoals artikelen uit vakbladen, documentaires, enzovoort. In het critical review hoofdstuk wordt het eerder geïntroduceerde onderwerp in verband gebracht met relevante literatuur en repertoire.

\subsection*{TODO}

\begin{itemize}
    \item Markt en 'echte' wereld
    \begin{itemize}
        \item Critical review naar bestaande media (Phonograph, LP, tape, CD, MP3, Torrent/P2P, Streaming)
        \item Korte technische uitleg wat het is
        \item Disruptieve werking van een medium op zijn voorganger
        \item Welk probleem heeft het proberen op te lossen
    \end{itemize}
    \item Eigen werk, directe context
    \begin{itemize}
        \item Korte geschiedenis van de NichePlayer (vorige iteraties)
        \item TapTapes
    \end{itemize}
\end{itemize}

% Inleiding
In dit critical review zal ik een onderzoek doen naar bestaande media en diensten voor muziek distributie. Hierbij zal ik beschrijven wat de disruptieve werking was van het beschreven medium ten opzichte van zijn voorganger, en wat het probleem was wat het medium probeerde op te lossen. Deze analyse is relevant omdat het project van dit Supportive Narrative (de NichePlayer) als ultiem doel heeft een nieuw medium te worden.

% ---------------------------------------------------------------------------------------- %
\subsection{Analyse bestaande media}
In dit onderzoek kan in theorie terug worden gegaan tot in de oudheid. Het doorgeven en distribueren van muziek is een onderdeel van cultuur, en daarmee iets wat ons mens maakt. Om een kaderen te stellen wordt gefocust op technologieën uit de afgelopen 150 jaar, beginnende bij de fonograaf.

\subsubsection*{Fonograaf}
% Beschrijving medium
De fonograaf (phonograph in Engels) is een van de eerste distributie media die het geluid zelf draagt ten opzichte van een beschrijving van de muziek. De geluidsgolven worden opgeslagen door met een naald in een wassen rol te krassen. De diepte van de kras staat gelijk aan de uitslag van het membraan. Tijdens het afspelen wordt het proces omgedraaid. Een naald glijdt door de krassen en brengt hiermee het membraan in beweging. Het geluid wordt vervolgens versterkt door een hoorn.

% Klinkende muziek in huis zonder instrument
Door de uitvinding van de fonograaf werd het mogelijk om klinkende muziek in huis te hebben, zonder dat hier een instrument voor bespeeld hoefde te worden. Hiermee is het dus ook mogelijk om een opname van een heel orkest af te spelen binnen huis, iets wat voor de uitvinding niet kon vanwege de ruimte.

% Minder focus op eigen interpretatie
Mensen konden hiermee ook 'perfecte' versies van de muziek beluisteren. In plaats van eigen interpretaties van bladmuziek (niet alles kan immers beschreven worden op papier) wordt geluisterd naar een enkele uitvoering van een artiest.

% Beperking: maximale speeltijd
Een beperking ten opzichte van zijn de voorgangers is dat de fonograaf (en opvolgende iteraties) een maximale speeltijd hadden, waar geschreven muziek in principe een oneindige lengte kan hebben. 

% NOTES
% - Opvolger op live muziek en zelf spelen
% - Muziek 'hoe het zou moeten klinken'
% - Redelijk goedkope manier om (klinkende) muziek in je huis te krijgen
% - Beperking in speeltijd

\subsubsection*{Grammofoon en iteraties}
De grammofoon is een iteratie op de wasrol. In plaats van een ronde rol is het geluid bij de grammofoon opgeslagen op een platte plaat. De het schrijven en het afspelen werkt bij de eerste iteraties hetzelfde als bij de grammofoon.

In de latere iteraties van de grammofoonspeler werd meer electronica verwerkt waardoor de kwaliteit en volume van het geluid beter werd. Daarnaast werd het ook mogelijk om de muziek in stereo af te spelen.

Een groot voordeel van de grammofoon is het medium waarop de muziek is opgeslagen: een plaat ten opzichte van de wasrol. Platen zijn het makkelijker te massa-produceren omdat ze gedrukt kunnen worden vanuit een mal. Een wasrol moet individueel worden beschreven en kan hierdoor niet makkelijk in massa geproduceerd worden. Platen zijn daarnaast ook makkelijker in opslag omdat ze plat zijn, en door het materiaal waarvan ze gemaakt zijn blijven ze langer goed.

\begin{itemize}
    \item Na wasrol platte plaat van vinyl
    \item Vinyl makkelijker in opslag
    \item Vinyl makkelijker in (massa) productie
    \item Veel iteraties: langere platen, stereo, kwaliteit verbetering, etc.
\end{itemize}

\subsubsection*{Tape}
% Beschrijving medium
De Tape is wordt het geluid opgeslagen op een magnetisch geladen strip.

\begin{itemize}
    \item Kopiëren (met kwaliteitsverlies)
    \item Cultuur => Mixtapes
    \item Opnemen vanuit andere media (piracy!)
\end{itemize}

\subsubsection*{CD}
\begin{itemize}
    \item Hoge kwaliteit
    \item Geen kwaliteit verlies na lange tijd opslag
\end{itemize}

\subsubsection*{Internet (Torrents en Peer2Peer)}
\begin{itemize}
    \item Onbeperkt kopiëren zonder kwaliteitsverlies
    \item Illegale onbeperkte toegang
    \item Downloaden lossen nummers
\end{itemize}

\subsubsection*{Streaming}
\begin{itemize}
    \item Legale onbeperkte toegang
    \item Album minder belangrijk -> playlists
\end{itemize}

% ---------------------------------------------------------------------------------------- %
\subsection{Concurrerende projecten}
\subsubsection*{TapTapes}
De NichePlayer is niet alleen in zijn soort. Er zijn veel vergelijkbare producten die het in de inleiding benoemde probleem proberen op te lossen. Een van deze diensten is TapTapes, een start-up uit Saarbrücken, Duitsland. Toevallig is dit het bedrijf waar ik in mijn bachelor stage heb gelopen. Het concept van de NichePlayer bestond toen al enkele jaren, maar heeft door de stage meer voor gekregen.
